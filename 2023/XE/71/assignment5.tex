\iffalse
\let\negmedspace\undefined
\let\negthickspace\undefined
\documentclass[journal,12pt,twocolumn]{IEEEtran}
\usepackage{cite}
\usepackage{amsmath,amssymb,amsfonts,amsthm}
\usepackage{algorithmic}
\usepackage{graphicx}
\usepackage{textcomp}
\usepackage{xcolor}
\usepackage{txfonts}
\usepackage{listings}
\usepackage{enumitem}
\usepackage{mathtools}
\usepackage{gensymb}
\usepackage{comment}
\usepackage[breaklinks=true]{hyperref}
\usepackage{tkz-euclide} 
\usepackage{listings}
\usepackage{gvv}                                        
\def\inputGnumericTable{}                                 
\usepackage[latin1]{inputenc}                                
\usepackage{color}                                            
\usepackage{array}                                            
\usepackage{longtable}                                       
\usepackage{calc}                                             
\usepackage{multirow}                                         
\usepackage{hhline}                                           
\usepackage{ifthen}                                           
\usepackage{lscape}

\newtheorem{theorem}{Theorem}[section]
\newtheorem{problem}{Problem}
\newtheorem{proposition}{Proposition}[section]
\newtheorem{lemma}{Lemma}[section]
\newtheorem{corollary}[theorem]{Corollary}
\newtheorem{example}{Example}[section]
\newtheorem{definition}[problem]{Definition}
\newcommand{\BEQA}{\begin{eqnarray}}
 \newcommand{\EEQA}{\end{eqnarray}}
\newcommand{\define}{\stackrel{\triangle}{=}}
\theoremstyle{remark}
\newtheorem{rem}{Remark}
\begin{document}
 \bibliographystyle{IEEEtran}
 \vspace{3cm}
 \title{\textbf{XE 71}}
 \author{EE23BTECH11048-Ponugumati Venkata Chanakya$^{*}$% <-this % stops a space
 }
 \maketitle
 \newpage
 \bigskip
 \renewcommand{\thefigure}{\theenumi}
 \renewcommand{\thetable}{\theenumi}
 \textbf{QUESTION:}
 A spring mass system is shown in the figure . Take the value of acceleration  due to gravity as $g=9.81m/s^2$.The static deflection due to weight and the time period of the oscillations,respectively,are\\
 \begin{figure}[h!]
    \centering
    \includegraphics[width = \columnwidth]{2023/XE/71/figs/xe_71_f1.png}
\end{figure}
\hfill{(GATE2023 XE)}\\
\solution
\fi
\begin{enumerate}
    \item Static deflection due to weight(sdw)\\
    let x be sdw.\\0
    At mean position in equilibrium\\
    \ref{XE_71.t1}
    \begin{align}
        Mg&=kx\\
        x&=39.24cm
    \end{align}
     \item Time period of oscillation\\
     \begin{align}
           F&=-kx\\
           m\brak{\frac{d^2x}{dt^2}}&=-kx
     \end{align}
      Initial Conditions be at extreme  point of SHM
      \begin{align}
       x(0)&=0.3924 \label{XE_71.5}\\
       \frac{dx}{dt}&=0 \text{ at } t=0 \text{ (released from rest)} \label{XE_71.6}
        \end{align} 
     Taking Laplace transform:
     \begin{align}
     m(s^2X(s)-sx(0)-mx'(0))+kX(s)&=0
     \end{align}
     \begin{align}
      X(s) &= \frac{x(0)ms+mx'(0)}{ms^2 + k}\\
      X(s)&=x(0)\frac{s}{s^2+\frac{k}{m}}+\brak{x'(0)\sqrt{\frac{k}{m}}}\frac{\sqrt{\frac{k}{m}}}{s^2+\brak{\sqrt{\frac{k}{m}}}^2}
      \end{align}
     Taking Inverse Laplace Transform:\\
     \begin{align}
     x(t)&=x(0)\cos\left(\sqrt{\frac{k}{m}}t\right)+\brak{x'(0)\sqrt{\frac{k}{m}}}\sin \brak{\sqrt{\frac{k}{m}}t}\\
     \end{align}
     Using \ref{XE_71.5} and \ref{XE_71.6} 
     \begin{align}
      x(t) &= 0.3924 \cos\left(\sqrt{\frac{k}{m}}t\right)\\
      x(t)&=39.24 \sin\brak{5t+\frac{\pi}{2}} \text{ cm}
     \end{align}
    The static deflection due to weight and the time period of the oscillations,respectively are $39.24$ cm and $\frac{2\pi}{5}$ s
\end{enumerate}
 \begin{figure}[h!]
    \centering
    \includegraphics[width = \columnwidth]{2023/XE/71/figs/xe_71_f2.png}
\end{figure}
 \begin{table}[!ht]
    \centering
             \begin{tabular}{|c|c|c|} 
      \hline
\textbf{Variable}& \textbf{Description}& \textbf{Value}\\\hline
         $M$& weight of block &$4$ kg\\\hline
          $K$ & spring constant & $100\frac{N}{m}$  \\\hline
          $x$& Static deflection due to weight&$39.24$ cm\\\hline  
          $x(t)$& Displacement of particle from mean position at time t  & none \\\hline
          $x(0)$& Initial Displacement of particle from mean position  & $39.24$cm \\\hline
           $x'(t)$& velocity of particle & none \\\hline
          $x'(0)$& initial velocity of particle & $0$ \\\hline
    \end{tabular}

    \caption{input parameters}
     \label{XE_71.t1}
\end{table}
%\end{document}

