\begin{enumerate}[label=\thechapter.\arabic*,ref=\thechapter.\theenumi]
\item
For the circuit given below, choose the angular frequency $ \omega_0$ at which voltage across capacitor has maximum amplitude?
\begin{figure}[h!]
    \includegraphics[width = 0.5\columnwidth]{2023/BM/16/figs/c_fig1.pdf}
    \caption{circuit }
    \centering
    \label{fig: bm_16_fig_1}
\end{figure}
\begin{enumerate}
    \item[(A)] 1000
    \item[(B)] 100
    \item[(C)] 1
    \item[(D)] 0   
\end{enumerate}
\hfill(GATE BM 2023 Question 16)\\

\solution
\input{2023/BM/16/asnmt3.tex}
\newpage
\item
In the following circuit, the switch S is open for $t < 0$ and closed for $t \ge 0$.
What is the steady state voltage (in Volts) across the capacitor when the switch is closed?
\begin{figure}[h!]
    \includegraphics[width = 0.7\columnwidth]{2023/BM/30/figs/c_fig1.pdf}
    \caption{circuit }
    \centering
    \label{fig:bm_30_fig_1}
\end{figure}
\hfill(GATE BM 2023 Question 30)\\
\item 
A finite impulse response (FIR) filter has only two non-zero samples in its impulse response $h[n]$, namely $h[0] = h[1] = 1$. The Discrete Time Fourier Transform (DTFT) of $h[n]$ equals $H(e^{j\omega})$, as a function of the normalized angular frequency $\omega$. For the range $\abs{\omega} \leq \pi$, $\abs{H(e^{j\omega})}$ is equal to
\begin{enumerate}
	\item[(A)] $2\abs{\cos(\omega)}$
	\item[(B)] $2\abs{\sin(\omega)}$
	\item[(C)] $2\abs{\cos(\frac{\omega}{2})}$
	\item[(D)] $2\abs{\sin(\frac{\omega}{2})}$
\end{enumerate}
\hfill(GATE BM 2023 Question 17) \\
\item
For the circuit shown,if $i=\sin 1000t$, the instantaneous value of the Thevenin's voltage(in volts) across the terminals a anb b at time t=5ms is\\[2pt]

\begin{circuitikz}[american voltages,american currents]
    % Draw the circuit components
    \draw (0,0) -- (2,0);
    \draw (2,2) to [resistor,l=$10\Omega$] (2,4);
    \draw (2,4) -- (0,4);
    \draw (2,0) to [capacitor,l=$-j10\Omega$,-,i_=$i_x$] (2,2);
    \draw (2,0) -- (5,0);
    \draw (5,0) to[inductor,l=$j10\Omega$] (5,2);
    \draw (5,2) to [resistor,l=$10\Omega$] (5,4);
  \draw (5,4) to [cV,l^=$4i_x$,invert] (2,4);
  \draw (5,4) -- (6,4);
  \draw (6,4) to[I,l=$\sin 1000t$,invert] (6,0);
  \draw (6,0) -- (5,0);
   \node[circle,fill=black,inner sep=1.5pt,label=above:a] at (0,0) {};
    \node[circle,fill=black,inner sep=1.5pt,label=above:b] at (0,4) {};
    \end{circuitikz}
    \hfill(GATE EE 2023 Question 51) \\
    \pagebreak

    \item In the circuit shown ,$\omega=100\pi\text{rads/s}$, R1=R2=$2.2\Omega$ and L=$7\text{mH}$. the capacitance $\text{C}$ for which $Y_{in}$ is purely real is  $\text{mF}$ \\
	\begin{center}
	\begin{circuitikz} \centering \draw 
		(0,4) to[sinusoidal voltage source, l=$V_{0}$cos($\omega$t)] (0,0)
		(0,4) to[short] (4,4)
		(4,4) to[resistor, l=$R_1$ ] (4,2)
		(4,2) to[inductor, l= $\text{L} $] (4,0) to[short ] (0,0)
		(8,4)  to[short] (4,4)
		(8,4) to[resistor, l= $R_2$] (8,2) to[capacitor,l=$\text{C}$] (8,0) to (4,0);
	\end{circuitikz}
	\end{center}
\hfill(GATE IN 2023 Q46)\\
\solution

\pagebreak
\item An input voltage in the form of a square wave of frequency $1\, kHz$ is given to a circuit, which results in the output shown schematically below. Which one of the following options is the CORRECT representation of the circuit? \hfill(GATE PH 2023 Q37)
\begin{figure}[!h]
    \centering
    \includegraphics[width = 0.6\columnwidth]{2023/PH/37/figs/question.png}
    \caption{}
\end{figure}

\begin{enumerate}[label = (\alph*)]
    \item 
    \begin{figure}[!h]
        \centering
	    \resizebox{0.2\textwidth}{!}{\input{2023/PH/37/figs/optA}}
	\label{optA_gate.ph.23.37}
    \end{figure}

    \item 
    \begin{figure}[!h]
        \centering
        \resizebox{0.2\textwidth}{!}{\input{2023/PH/37/figs/optB}}
        \label{optB_gate.ph.23.37}
    \end{figure}

    \item 
    \begin{figure}[!h]
        \centering
        \resizebox{0.2\textwidth}{!}{\input{2023/PH/37/figs/optC}}
        \label{optC_gate.ph.23.37}
    \end{figure}

    \item 
    \begin{figure}[!h]
        \centering
        \resizebox{0.2\textwidth}{!}{\begin{circuitikz}
    \draw(0, 0) to[short,*-*] ++ (4,0);
\draw (0,2) to[R, l = $5k\Omega$, *-] ++ (3,0) coordinate(a);
\draw (a) to[short,-*] ++ (1,0);
\draw (a) to[C,l_=$1\mu F$,*-*] ++(0,-2);

% Voltage labels
\draw (0,2) to[open,l_=V$_{in}$] ++(0,-2);
\draw (4,2) to[open,l=V$_{out}$] ++(0,-2);
\end{circuitikz}

}
        \label{optD_gate.ph.23.37}
    \end{figure}
\end{enumerate} \hfill(GATE 2023 PH 37)
\solution

\pagebreak
\pagebreak

\item In the circuit shown below, switch S was closed for long time. If the switch is opened at $t=0$, the  maximum magnitude of the voltage $V_R$ , in volts is (rounded off to the nearest integer)\hfill{(GATE 2023 EC 35)}\\
\begin{figure}[h!]
    \centering
    \includegraphics[width=1\linewidth]{2023/EC/35/figs/gate.png}
    \caption{ }
\end{figure}
\solution
\end{enumerate}
