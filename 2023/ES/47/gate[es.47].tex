 \iffalse
\let\negmedspace\undefined
\let\negthickspace\undefined
\documentclass[journal,12pt,twocolumn]{IEEEtran}
\usepackage{xparse}
\usepackage{cite}
\usepackage{amsmath,amssymb,amsfonts,amsthm}
\usepackage{algorithmic}
\usepackage{graphicx}
\usepackage{textcomp}
\usepackage{xcolor}
\usepackage{txfonts}
\usepackage{listings}
\usepackage{enumitem}
\usepackage{mathtools}
\usepackage{gensymb}
\usepackage{comment}
\usepackage[breaklinks=true]{hyperref}
\usepackage{tkz-euclide} 
\usepackage{listings}
\usepackage{gvv}
\def\inputGnumericTable{}                                 
\usepackage[latin1]{inputenc}                                
\usepackage{color}                                            
\usepackage{array}                                            
\usepackage{longtable}                                       
\usepackage{calc}                                             
\usepackage{multirow}                                         
\usepackage{hhline}                                           
\usepackage{ifthen}                                           
\usepackage{lscape}

\newtheorem{theorem}{Theorem}[section]
\newtheorem{problem}{Problem}
\newtheorem{proposition}{Proposition}[section]
\newtheorem{lemma}{Lemma}[section]
\newtheorem{corollary}[theorem]{Corollary}
\newtheorem{example}{Example}[section]
\newtheorem{definition}[problem]{Definition}
\newcommand{\BEQA}{\begin{eqnarray}}
\newcommand{\EEQA}{\end{eqnarray}}
\newcommand{\define}{\stackrel{\triangle}{=}}
\theoremstyle{remark}
\newtheorem{rem}{Remark}
\begin{document}

\bibliographystyle{IEEEtran}
\vspace{3cm}

\title{GATE-ES.47}
\author{EE23BTECH11046 - Poluri Hemanth$^{*}$}
\maketitle
\textbf{Question:}Second order ordinary differential equation $\frac{d^2y}{dx^2}-\frac{dy}{dx}-2y=0$ has values 
$y=2$ and$\frac{dy}{dx}=1$ at $x=0$.The value of $y$ at $x=1$ is?($round\; off\;\: to\;\: three\;\: decimal\;\: places$)
 \\ \hfill[GATE-ES 2023]\\
\textbf{Solution:}\\
\fi
We convert given second order differential equation to s domain using Laplace transform and solve for $Y(s)$ and take inversion to get $y(x)$.
\begin{table}[h!]
    % Change address in github
	\begin{table}[ht!]
\centering
\begin{tabular}{ |c|c| } 
 \hline
Symbols & Description \\
\hline
 $X_1(s)$ & Laplace transform of $x_1(t)$ \\
 \hline
 $X_2(s)$ & Laplace transform of $x_2(t)$\\
\hline
 $u(t)$ & Unit step function\\
\hline
\end{tabular}
\caption{Parameters, Descriptions}
\label{table:ee25-tab2}
\end{table}




        \caption{Parameters}
        \label{tab:es.47}
\end{table}


\begin{align}
    \frac{d^2y}{dx^2}-\frac{dy}{dx}-2y&\Large\xleftrightarrow{\mathcal{L}}s^2Y(s)-sy(0)-y'(0)-sY(s)+y(0)-2Y(s)\\
	Y(s)\left(s^2-s-2\right)&=2s-1\\
    \Rightarrow Y(s)&=\frac{2s-1}{s^2-s-2}\\
    \Rightarrow Y(s)&=\frac{1}{s-2}+\frac{1}{s+1}
\end{align}
For inversion of $Y(s)$ in partial fractions-
\begin{align}
	&\frac{b}{s+a}\Large\xleftrightarrow{\mathcal{L}^{-1}}be^{-ax}u(x)\label{inv}
\end{align}
Where b, a are real numbers, we invert $Y(s)$ to get $y(x)$:-\\
From \eqref{inv}
\begin{align}
	&Y(s)\Large\xleftrightarrow{\mathcal{L}^{-1}} y(x)u(x)
\end{align}
\begin{align}
	y(x)&=\left(e^{2x}+e^{-x}\right)u(x)\\
   \Rightarrow y(1)&=7.757
\end{align}\\
\begin{figure}[h!]
    \centering
    \includegraphics[width=1\linewidth]{2023/ES/47/figures/gate.png}
	\caption{Plot of y(x)}
    \label{fig:enter-label}
\end{figure}






%\end{document}
