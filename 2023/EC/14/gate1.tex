\iffalse
\documentclass[journal,12pt,twocolumn]{IEEEtran}
\usepackage{cite}
\usepackage{amsmath,amssymb,amsfonts,amsthm}
\usepackage{algorithmic}
\usepackage{graphicx}
\usepackage{textcomp}
\usepackage{xcolor}
\usepackage{txfonts}
\usepackage{listings}
\usepackage{enumitem}
\usepackage{mathtools}
\usepackage{gensymb}
\usepackage{comment}
\usepackage[breaklinks=true]{hyperref}
\usepackage{tkz-euclide}
\usepackage{gvv}
\def\inputGnumericTable{}
\usepackage[latin1]{inputenc}
\usepackage{color}
\usepackage{array}
\usepackage{longtable}
\usepackage{calc}
\usepackage{multirow}
\usepackage{hhline}
\usepackage{tikz}
\usetikzlibrary{decorations.pathreplacing}

\newtheorem{theorem}{Theorem}[section]
\newtheorem{problem}{Problem}
\newtheorem{proposition}{Proposition}[section]
\newtheorem{lemma}{Lemma}[section]
\newtheorem{corollary}[theorem]{Corollary}
\newtheorem{example}{Example}[section]
\newtheorem{definition}[problem]{Definition}
\newcommand{\BEQA}{\begin{eqnarray}}
\newcommand{\EEQA}{\end{eqnarray}}
\newcommand{\define}{\stackrel{\triangle}{=}}
\theoremstyle{remark}
\newtheorem{rem}{Remark}

\begin{document}

\bibliographystyle{IEEEtran}
\vspace{3cm}

\title{GATE 2023-EC}
\author{EE23BTECH1205 - Avani Chouhan$^{*}$% <-this % stops a space
}
\maketitle
\newpage
\bigskip

\renewcommand{\thefigure}{\theenumi}
\renewcommand{\thetable}{\theenumi}

\vspace{3cm}
\textbf{Question : 14} \\
The value of the contour integral, $\oint_C \frac{z + 2}{z^2 + 2z + 2} \, dz$, where the contour $C$ is $\{ z : |z + 1 - \frac{3}{2}i| = 1 \}$, taken in the counter clockwise direction, is \\

\begin{enumerate}
  \item[(A)] $-\pi(1+j) $
  \item[(B)] $\pi(1+j)$
  \item[(C)] $\pi(1-j) $
  \item[(D)] $-\pi(1-j)$
\end{enumerate}

\hfill{(GATE ST 2023)}\\
\textbf{Solution:}
\fi
\begin{align}
I = \oint_{C} \frac{z + 2}{z^2 + 2z + 2} \, dz; \quad C = \left| z + 1 - \frac{3}{2}i \right| = 1 
\end{align}

The poles are given by $(z + 1)^2 + 1 = 0$\\
\begin{align}
    z + 1 &= \pm \sqrt{-1}
\end{align}
$z = -1 + j$, $z = -1 - j$\\
where $-1 - i$ lies outside $C$ and $z = (-1, 1)$ lies inside $C$, by the Residue Theorem:
\begin{align}
\oint_C f(z) \, dz &= 2\pi i \, \text{Res}(f(z), z=-1+j) \\
&= 2\pi i \left( \frac{z+2}{2(z+1)} \right) \bigg|_{z=-1+i} \\
&= 2\pi i \left( \frac {-1+j+2} {2(-1+j+1)} \right) \\
&= \pi (1+j).
\end{align}

Therefore, the correct answer is option (B).

\begin{figure}[htbp]
    \centering
    \begin{tikzpicture}[scale=1.5]
        \draw[->] (-1.5,0) -- (1.5,0) node[right] {$\Re$};
        \draw[->] (0,-2) -- (0,2) node[above] {$\Im$};
        \draw (0,0) circle [radius=1];
        \node[circle,fill,inner sep=1.5pt,label=below left:{$-1-j$}] at (-1,-1) {};
        \node[circle,fill,inner sep=1.5pt,label=above left:{$-1+j$}] at (-1,1) {};
        \draw[thick,->] (-1,0) arc [start angle=180, end angle=120, radius=1];
        \node[above left] at (-0.8,0.4) {$C$};
    \end{tikzpicture}
    \caption{Contour $C$ and poles}
    \label{fig:contour}
\end{figure}

%\end{document}

