\begin{enumerate}[label=\thechapter.\arabic*,ref=\thechapter.\theenumi]
\item Let a frequency modulated (FM) signal : $ x(t) = A \cos(\omega_c t + k_f \int_{-\infty}^{t} m(\lambda) d\lambda)$ , where $ m(t) $is a message signal of bandwidth $ W $. It is passed through a non-linear system with output $y(t) = 2x(t) + 5(x(t))^2 $.
Let $B_T $denote the FM bandwidth. The minimum value of $ \omega_c $ required to recover $ x(t) $ from $ y(t) $ is:\\
\begin{enumerate}[label = (\Alph*)]
\item $B_T + W$ \\
\item $\dfrac{3}{2} B_T$ \\
\item $2B_T + W$ \\
\item $\dfrac{5}{2} B_T$ \\
\end{enumerate}

\solution
\newpage

\item Let an input $x[n]$ having discrete-time Fourier transform
$X(e^{j\Omega}) = 1 - e^{-j\Omega} + 2e^{-3j\Omega}$
be passed through an LTI system. The frequency response of the LTI system is 
$H(e^{j\Omega}) = 1 - \frac{1}{2} e^{-2j\Omega}$
The output $y[n]$ of the system is \\ \hfill(GATE EC 2023)
\solution 
\input{2023/EC/48/ec48.tex}
\newpage
\item The Fourier transform $X(\omega)$ of $x(t) = e^{-t^2}$ is\\
Note:$\int_{-\infty}^{\infty} e^{-y^2} \,dy = \sqrt{\pi}$ \\  
A) $\sqrt{\pi} e^{\frac{\omega^2}{2}}$ \\
B) $\frac{e^{\frac{-\omega^2}{4}}}{2\sqrt{\pi}}$ \\
C) $\sqrt{\pi} e^{\frac{-\omega^2}{4}}$ \\
D) $\sqrt{\pi} e^{\frac{-\omega^2}{2}}$\\
\hfill Gate 2023 EC Question 28
\solution
\let\negmedspace\undefined
\let\negthickspace\undefined
\documentclass[journal,12pt,twocolumn]{IEEEtran}
\usepackage{cite}
\usepackage{amsmath,amssymb,amsfonts,amsthm}
\usepackage{algorithmic}
\usepackage{graphicx}
\usepackage{textcomp}
\usepackage{xcolor}
\usepackage{txfonts}
\usepackage{listings}
\usepackage{enumitem}
\usepackage{mathtools}
\usepackage{gensymb}
\usepackage{comment}
\usepackage[breaklinks=true]{hyperref}
\usepackage{tkz-euclide} 
\usepackage{listings}                                   
\def\inputGnumericTable{}                                 
\usepackage[latin1]{inputenc}                                
\usepackage{color}                                            
\usepackage{array}                                            
\usepackage{longtable}                                       
\usepackage{calc}                                             
\usepackage{multirow}                                         
\usepackage{hhline}                                           
\usepackage{ifthen}                                           
\usepackage{lscape}
\newtheorem{theorem}{Theorem}[section]
\newtheorem{problem}{Problem}
\newtheorem{proposition}{Proposition}[section]
\newtheorem{lemma}{Lemma}[section]
\newtheorem{corollary}[theorem]{Corollary}
\newtheorem{example}{Example}[section]
\newtheorem{definition}[problem]{Definition}
\newcommand{\BEQA}{\begin{eqnarray}}
\newcommand{\EEQA}{\end{eqnarray}}
\newcommand{\define}{\stackrel{\triangle}{=}}
\newcommand{\brak}[1]{\langle #1 \rangle}
\theoremstyle{remark}
\newtheorem{rem}{Remark}

\begin{document}
\bibliographystyle{IEEEtran}
\vspace{3cm}
\title{\textbf{GATE 2023 EC}}
\author{EE23BTECH11023-ABHIGNYA GOGULA}
\maketitle
\newpage
\bigskip
\renewcommand{\thefigure}{\theenumi}
\renewcommand{\thetable}{\theenumi}
\textbf{Question28:}
\\
 The Fourier transform $X(\omega)$ of $x(t) = e^{-t^2}$ is\\
Note:$\int_{-\infty}^{\infty} e^{-y^2} \,dy = \sqrt{\pi}$ \\  
A) $\sqrt{\pi} e^{\frac{\omega^2}{2}}$ \\
B) $\frac{e^{\frac{-\omega^2}{4}}}{2\sqrt{\pi}}$ \\
C) $\sqrt{\pi} e^{\frac{-\omega^2}{4}}$ \\
D) $\sqrt{\pi} e^{\frac{-\omega^2}{2}}$\\
\hfill Gate 2023 EC Question 28
\end{document}

\newpage

 \item Let $x(t) = 10 \cos(10.5 \omega t)$ be passed through an LTI system with impulse response $h(t) = \pi\left(\frac{\sin(\omega t)}{\pi t}\right)^2 \cos(10 \omega t)$ . The output of the system is:\\ \hfill(GATE EC 2023)
 \solution
 \newpage
 
 \item Q27) Let m\brak{\text{t}} be a strictly band-limited signal with bandwidth B and energy E. Assuming $\omega_0$ = 10B, the energy in the signal $\text{m}\brak{\text{t}}\text{cos}\brak{\omega_0\text{t}}$\\[1ex]
\brak{\text{A}}\ $\frac{\text{E}}{4}$\\[1ex]
\brak{\text{B}}\ $\frac{\text{E}}{2}$\\[1ex]
\brak{\text{C}}\ \text{E}\\[1ex]
\brak{\text{D}}\ 2\text{E} \qquad\qquad\qquad\quad\qquad\qquad\qquad\qquad\brak{\text{GATE EC 2023}}

\solution
\input{2023/EC/27/gate,2k23.tex}
\newpage

\item The following function is defined over the interval $[-L,L]:$
    $$f\brak{x}=px^4+qx^5$$
It is expressed as a Fourier series,
    $$f\brak{x}=a\brak{0}+\sum_{n=1}^{\infty}\cbrak{a\brak{n}\sin\brak{\frac{\pi x}{L}}+b\brak{n}\cos\brak{\frac{\pi x}{L}}}$$

which options amongst the following are true?
\begin{enumerate}[label=(\alph*)]
    \item $a\brak{n}$, $n=1,2,..,\infty$ depend on $p$
    \item $a\brak{n}$, $n=1,2,..,\infty$ depend on $q$
    \item $b\brak{n}$, $n=1,2,..,\infty$ depend on $p$
    \item $b\brak{n}$, $n=1,2,..,\infty$ depend on $q$
\end{enumerate}
\hfill(GATE 2023 CE Question 25)\\
\solution
\iffalse
\let\negmedspace\undefined
\let\negthickspace\undefined
\documentclass[journal,12pt,twocolumn]{IEEEtran}
\usepackage{cite}
\usepackage{amsmath,amssymb,amsfonts,amsthm}
\usepackage{algorithmic}
\usepackage{graphicx}
\usepackage{textcomp}
\usepackage{xcolor}
\usepackage{txfonts}
\usepackage{listings}
\usepackage{enumitem}
\usepackage{mathtools}
\usepackage{float}
\usepackage{gensymb}
\usepackage{comment}
\usepackage[breaklinks=true]{hyperref}
\usepackage{tkz-euclide} 
\usepackage{listings}
\usepackage{gvv}                                        
\def\inputGnumericTable{}                                 
\usepackage[latin1]{inputenc}                                
\usepackage{color}                                            
\usepackage{array}          
\usetikzlibrary{positioning, arrows.meta}
\usepackage{longtable}                                       
\usepackage{calc}                                             
\usepackage{multirow}                                         
\usepackage{hhline}                                           
\usepackage{ifthen}                                           
\usepackage{lscape}
\usepackage{amsmath}
\newtheorem{theorem}{Theorem}[section]
\newtheorem{problem}{Problem}
\newtheorem{proposition}{Proposition}[section]
\newtheorem{lemma}{Lemma}[section]
\newtheorem{corollary}[theorem]{Corollary}
\newtheorem{example}{Example}[section]
\newtheorem{definition}[problem]{Definition}
\newcommand{\BEQA}{\begin{eqnarray}}
\newcommand{\EEQA}{\end{eqnarray}}
\newcommand{\define}{\stackrel{\triangle}{=}}
\theoremstyle{remark}
\newtheorem{rem}{Remark}
\begin{document}

\bibliographystyle{IEEEtran}
\title{GATE-EE-Q14}
\author{EE23BTECH11015 - DHANUSH V NAYAK$^{*}$% <-this % stops a space
}
\maketitle
\newpage
\bigskip
\renewcommand{\thefigure}{\arabic{figure}}
\renewcommand{\thetable}{\theenumi}
\textbf{Question:}Consider a unity-gain negative feedback system consisting of the plant $G\brak{s}$  and a proportional-integral controller. Let the proportional gain and integral
gain be 3 and 1, respectively. For a unit step reference input, the final values of the
controller output and the plant output, respectively, are
\begin{align}
    G\brak{s} = \frac{1}{\brak{s-1}} \notag
\end{align}
\solution 
\fi
\begin{table}[H]
\centering
\renewcommand\thetable{1}
\setlength{\extrarowheight}{9pt}
\resizebox{0.5\textwidth}{!}{
\begin{tabular}{|c|c|c|}
\hline
\textbf{Parameter} & \textbf{Description} & \textbf{Value} \\ \hline
$K_{p}$ & Proportional Gain & 3  \\ \hline
$K_{i}$ & Integral Gain &1 \\ \hline
$r\brak{t}$& Reference Input & $u\brak{t}$ \\ \hline 
$w\brak{t}$& Controller Output & $?$ \\ \hline 
$y\brak{t}$ & Plant Output & $?$ \\ \hline
$e\brak{t}$ & Error Input & $r\brak{t}-y\brak{t}$ \\ \hline
\end{tabular}}
\caption{Parameter Table}
\label{tab:gate_ee_Q14}
\end{table}

From the\figref{fig:gate_ee_Q14_blockdiagram}:
\begin{align}
    E\brak{s}&= U\brak{s} - Y\brak{s}\label{eq:gate_ee_Q14.1}\\
W\brak{s} &= 3E\brak{s} + \frac{1}{s}E\brak{s}\label{eq:gate_ee_Q14.2}\\
    Y\brak{s} &= G\brak{s}W\brak{s} \label{eq:gate_ee_Q14.3}
\end{align}
Some results:
\begin{align}
    tx\brak{t} &\system{L} -\frac{d{X\brak{s}}}{ds} \label{eq:laplace_diff_prop}\\
    e^{-at}x\brak{t} &\system{L} X\brak{s+a}\label{eq:laplace_timeshifting_prop}
\end{align}
By using \eqref{eq:laplace_diff_prop} and \eqref{eq:laplace_timeshifting_prop}:
\begin{align}
    e^{-t}u\brak{t} &\system{L} \frac{1}{s+1} ,  Re\brak{s}>-1 \label{eq:gate_ee_Q14result.1}\\
    t e^{-t}u\brak{t} &\system{L} \frac{1}{\brak{s+1}^2},  Re\brak{s}>-1 
 \label{eq:gate_ee_Q14result.2}
\end{align}

\begin{figure}[H]
    \resizebox{0.9\textwidth}{!}{\tikzset{
    block/.style = {draw, fill=white, rectangle, minimum height=3em, minimum width=3em},
    tmp/.style  = {coordinate}, 
    minus/.style= {draw, fill=white, circle, node distance=1cm, append after command={\pgfextra \draw ($(\tikzlastnode.center) + (-0.15,0)$) -- ($(\tikzlastnode.center) + (0.15,0)$) node[above] {$-$}; \endpgfextra}},
    plus/.style= {draw, fill=white, circle, node distance=1cm, append after command={\pgfextra \draw ($(\tikzlastnode.center) + (-0.15,0)$) -- ($(\tikzlastnode.center) + (0.15,0)$) node[above] {$+$}; \endpgfextra}},
    input/.style = {coordinate},
    output/.style= {coordinate},
    pinstyle/.style = {pin edge={to-,thin,black}}
}


\begin{tikzpicture}[auto, node distance=2cm,>=latex]
    \node [input, name=rinput] (rinput) {};
    \node [minus, right of=rinput] (sum1) {};
    
    \node [block, right of=sum1] (controller) {$k_{p}=3$};
    \node [block, above of=controller, node distance=2cm] (up) {$\frac{k_{i}}{s}=\frac{3}{s}$};
    
    \node [plus, right of=controller, node distance=2cm] (sum2) {};
    \node [block, right of=sum2, node distance=3.5cm] (system) {$G\brak{s}=\frac{1}{\brak{s-1}}$};
    \node [output, right of=system, node distance=2cm] (output) {};
    \node [tmp, below of=controller] (tmp1) {$H(s)$};

    \draw [->] (rinput) -- node[below]{$r\brak{t}$} (sum1);
    \draw [->] (sum1) -- node[name=z,anchor=north,fill=white,circle,inner sep=1pt]{$e\brak{t}$} (controller);
    \draw [->] (controller) -- (sum2);
    \draw [->] (sum2) -- node[above, pos=0.8]{$w\brak{t}$} (system);
    \draw [->] (system) -- node [name=y] {$y\brak{t}$} (output);
    \draw [->] (z) |- (up);
    \draw [->] (up) -| (sum2);
    \draw [->] (y) |- (tmp1) -| (sum1);
\end{tikzpicture}
}
    \caption{Block Diagram of System}
    \label{fig:gate_ee_Q14_blockdiagram}
\end{figure}
\begin{enumerate}
\item \textbf{Plant Output:}\\
From \eqref{eq:gate_ee_Q14.1} , \eqref{eq:gate_ee_Q14.2} and \eqref{eq:gate_ee_Q14.3}:
\begin{align}
    Y\brak{s} &=  \frac{3s+1}{s\brak{s+1}^2} ,  Re\brak{s}>-1 \label{eq:Y(s)}
\end{align}
Final Value Theorem:    
\begin{align}
    \lim_{t \to \infty} x\brak{t}&= \lim_{s \to 0} sX\brak{s}\label{eq:finalval_thm}
\end{align}
Using \eqref{eq:finalval_thm} on Y\brak{s}:
\begin{align}
     \lim_{t \to \infty} y\brak{t}&= \lim_{s \to 0} sY\brak{s}\\
                            &= 1
\end{align}

Taking partial fraction of \eqref{eq:Y(s)} :
\begin{align}
    Y\brak{s} &= \frac{1}{s} + \frac{2}{\brak{s+1}^2} - \frac{1}{s+1}
\end{align}
Using \eqref{eq:gate_ee_Q14result.1} and \eqref{eq:gate_ee_Q14result.2}:
\begin{align}
    \therefore y\brak{t} &= u\brak{t}+ 2t e^{-t}u\brak{t} - e^{-t}u\brak{t}
\end{align}
\item \textbf{Controller Output:}\\
From \eqref{eq:gate_ee_Q14.2}
\begin{align}
     W\brak{s} &= \frac{3}{s} + \frac{1}{s^2} - Y\brak{s}\brak{3+\frac{1}{s}}
\end{align}
Substituting \eqref{eq:Y(s)}
\begin{align}
    W\brak{s} &= \frac{\brak{s-1}\brak{3s+1}}{s\brak{s+1}^2} ,  Re\brak{s}>-1 \label{eq:W(s)}
\end{align}
Using \eqref{eq:finalval_thm} on W\brak{s}
\begin{align}
     \lim_{t \to \infty} w\brak{t}&= \lim_{s \to 0} sW\brak{s}\\
                            &= -1
\end{align}
Taking partial fraction of equation\eqref{eq:W(s)} :
\begin{align}
    W\brak{s} &= -\frac{1}{s} - \frac{4}{\brak{s+1}^2} + \frac{4}{s+1}
\end{align}
Using equations \eqref{eq:gate_ee_Q14result.1} and \eqref{eq:gate_ee_Q14result.2} and taking inverse lapalace transform:
\begin{align}
    w\brak{t} &= -u\brak{t}-4t e^{-t}u\brak{t} +4 e^{-t}u\brak{t}
\end{align}

\begin{figure}[H]
    \includegraphics[width=1\columnwidth]{2023/EE/14/figs/Plot of w(t).png}
    \caption{$w\brak{t}$ converges at -1.}
    \label{fig:w_t}
\end{figure}

\begin{figure}[H]
    \includegraphics[width=1\columnwidth]{2023/EE/14/figs/Plot of y(t).png}
    \caption{$y\brak{t}$ converges at +1}
    \label{fig:y_t}
\end{figure}

\end{enumerate}
%\end{document}


\newpage
\item A continuous real-valued signal $x\brak{t}$ has finite positive energy and $x\brak{t} = 0$, $\forall$ $t < 0$. From the list given below, select ALL the signals whose
continuous-time Fourier transform is purely imaginary.\\
\begin{enumerate}
\item$x\brak{t} + x\brak{-t}$
\item$x\brak{t} - x\brak{-t}$
\item$j\brak{x\brak{t} + x\brak{-t}}$
\item$j\brak{x\brak{t} - x\brak{-t}}$
\end{enumerate}
\hfill{(GATE IN 2023)}\\
\solution
\iffalse
\let\negmedspace\undefined
\let\negthickspace\undefined
\documentclass[journal,12pt,twocolumn]{IEEEtran}
\usepackage{cite}
\usepackage{amsmath,amssymb,amsfonts,amsthm}
\usepackage{algorithmic}
\usepackage{graphicx}
\usepackage{textcomp}
\usepackage{xcolor}
\usepackage{txfonts}
\usepackage{listings}
\usepackage{enumitem}
\usepackage{mathtools}
\usepackage{gensymb}
\usepackage{comment}
\usepackage[breaklinks=true]{hyperref}
\usepackage{tkz-euclide} 
\usepackage{listings}
\usepackage{gvv}                                        
\def\inputGnumericTable{}                                 
\usepackage[latin1]{inputenc}                                
\usepackage{color}                                            
\usepackage{array}                                            
\usepackage{longtable}                                       
\usepackage{calc}                                             
\usepackage{multirow}                                         
\usepackage{hhline}                                           
\usepackage{ifthen}                                           
\usepackage{lscape}

\newtheorem{theorem}{Theorem}[section]
\newtheorem{problem}{Problem}
\newtheorem{proposition}{Proposition}[section]
\newtheorem{lemma}{Lemma}[section]
\newtheorem{corollary}[theorem]{Corollary}
\newtheorem{example}{Example}[section]
\newtheorem{definition}[problem]{Definition}
\newcommand{\BEQA}{\begin{eqnarray}}
\newcommand{\EEQA}{\end{eqnarray}}
\newcommand{\define}{\stackrel{\triangle}{=}}
\theoremstyle{remark}
\newtheorem{rem}{Remark}
\begin{document}
\bibliographystyle{IEEEtran}
\vspace{3cm}
\title{\textbf{IN-2023}}
\author{EE23BTECH11210-Dhyana Teja Machineni$^{*}$% <-this % stops a space
}
\maketitle
\newpage
\bigskip

\textbf{QUESTION:}\\
A continuous real-valued signal $x\brak{t}$ has finite positive energy and $x\brak{t} = 0$, $\forall$ $t < 0$. From the list given below, select ALL the signals whose
continuous-time Fourier transform is purely imaginary.\\
\begin{enumerate}
\item$x\brak{t} + x\brak{-t}$
\item$x\brak{t} - x\brak{-t}$
\item$j\brak{x\brak{t} + x\brak{-t}}$
\item$j\brak{x\brak{t} - x\brak{-t}}$
\end{enumerate}
\hfill{(GATE IN 2023)}\\
\solution\\
\fi
\begin{table}[h]
         \renewcommand{\arraystretch}{1.5}
\begin{tabular}{|c|c|}
\hline
Parameter & Description  \\\hline
$x(t)$ & Continuous real valued signal  \\\hline
$t$ & time \\\hline
$f$ & frequency of the signal \\\hline
$Y(f)$& Fourier Transfom of $y(t)$\\\hline
\end{tabular}

         \caption{Variables and their descriptions}
     \end{table}\\
Fourier transform of a signal $y\brak{t}$\\
\begin{align}
\mathcal{F}\{y(t)\} &= Y\brak{f}\\
Y\brak{f}&=\int_{-\infty}^{\infty} y\brak{t} e^{-j 2\pi f t} \ dt\\
Y^*\brak{f}&=\int_{-\infty}^{\infty} y^*\brak{t} e^{j 2\pi f t} \ dt
\end{align}
Fourier transform is purely imaginary if $Y\brak{f}+Y^*\brak{f}=0$\\
\begin{enumerate}
\item $x\brak{t} + x\brak{-t}$\\
\begin{align}
 y\brak{t}&=x\brak{t} + x\brak{-t}\\
 y^*\brak{t}&=y\brak{t}\\
 y\brak{t}&=y\brak{-t}\\
 Y\brak{f}+Y^*\brak{f}&=\int_{-\infty}^{\infty} y\brak{t} e^{-j 2\pi f t} \ dt+\int_{-\infty}^{\infty} y^*\brak{t} e^{j 2\pi f t} \ dt\\
 &=2\int_{-\infty}^{\infty} y\brak{t} cos\brak{2 \pi ft} \ dt
\end{align}
$\therefore$ Fourier Transform is Purely real.\\
\item $x\brak{t} - x\brak{-t}$\\
\begin{align}
  y\brak{t}&=x\brak{t} - x\brak{-t}\\
  y^*\brak{t}&=y\brak{t}=-y\brak{-t}\\
  Y\brak{f}&=\int_{-\infty}^{\infty} y\brak{t} e^{-j 2\pi f t} \ dt\\
  Y^*\brak{f}&=-\int_{-\infty}^{\infty} y\brak{-t} e^{j 2\pi f t} \ dt\\
  &=-\int_{-\infty}^{\infty} y\brak{t} e^{-j 2\pi f t} \ dt\\
  Y\brak{f}+Y^*\brak{f}&=0
\end{align}
$\therefore$ Fourier Transform is purely imaginary.\\
\item $j\brak{x\brak{t} + x\brak{-t}}$
\begin{align}
  y\brak{t}&= j\brak{x\brak{t} + x\brak{-t}}\\
  y\brak{-t}&=y\brak{t}\\
  y^*\brak{t}&=-y(t)\\
  Y\brak{f}&=\int_{-\infty}^{\infty} y\brak{t} e^{-j 2\pi f t} \ dt\\
  Y^*\brak{f}&=-\int_{-\infty}^{\infty} y\brak{t} e^{j 2\pi f t} \ dt\\
  &=-\int_{-\infty}^{\infty} y\brak{t} e^{-j 2\pi f t} \ dt\\
  Y\brak{f}+Y^*\brak{f}&=0
\end{align}
$\therefore$ Fourier Transform is Purely imaginary.\\
\item $j\brak{x\brak{t} - x\brak{-t}}$
\begin{align}
  y\brak{t}&=j\brak{x\brak{t} - x\brak{-t}}\\
  y\brak{-t}&=-y\brak{t}\\
  y^*\brak{t}&=-y\brak{t}\\
  Y\brak{f}&=\int_{-\infty}^{\infty} y\brak{t} e^{-j 2\pi f t} \ dt\\
  Y^*\brak{f}&=-\int_{-\infty}^{\infty} y\brak{t} e^{j 2\pi f t} \ dt\\
  &=\int_{-\infty}^{\infty} y\brak{t} e^{-j 2\pi f t} \ dt\\
  Y\brak{f}+Y^*\brak{f}&=2\int_{-\infty}^{\infty} y\brak{t} e^{-j 2\pi f t} \ dt
  \end{align}
$\therefore$ Fourier Transform is not Purely imaginary.\\
\end{enumerate}
%\end{document}

\item Let $x_1(t) = u(t + 1.5) - u(t - 1.5)$ and $x_2(t)$ is shown in the figure below. For $y(t) = x_1(t) * x_2(t)$, the $\int_{-\infty}^{\infty} y(t) \, dt$ is \underline{\hspace{2cm}}.\\

\begin{figure}[htbp]
    \centering
    \includegraphics[width=0.5\textwidth]{2023/EC/58/figs/gatefig.png}
    \caption{Figure}
    \label{fig:graph}
\end{figure}

\hfill{(GATE IN 2023)}\\
\solution
\input{2023/EC/58/gate_ec_58.tex}
\pagebreak
\item Consider a discrete-time signal with period $N=5$. Let the discrete-time Fourier series (DTFS) representation be $ x[n] = \sum\limits_{k=0}^{4} a_k e^{\frac{jk2\pi n}{5}} $where $a_0=1$, $a_1=3j$, $a_2=2j$, $a_3=-2j$, $a_4=-3j$. The value of the sum $\sum\limits_{n=0}^{4}x[n] \sin\left(\frac{4\pi n}{5}\right) $is\\
(A) -10\\
(B) 10\\
(C) -2\\
(D) 2\\
\hfill Gate 2023 EC 47
\solution
\pagebreak

\item A continuous time, band-limited signal $x(t)$ has its Fourier transform described by:
\[ X(f) = \begin{cases} 
1 - \frac{|f|}{200} & \text{if } |f| \leq 200 \\
0 & \text{if } |f| > 200 
\end{cases} \]
The signal is uniformly sampled at a sampling rate of 600 Hz. The Fourier transform of the signal is $X_s(f)$. What is the value of $\frac{X_s(600)}{X_s(500)}$? \\\hfill{(GATE 2023 BM)}
\solution
\iffalse
\let\negmedspace\undefined
\let\negthickspace\undefined
\documentclass[journal,12pt,twocolumn]{IEEEtran}

\usepackage{cite}
\usepackage{amsmath,amssymb,amsfonts,amsthm}
\usepackage{graphicx}
\usepackage{textcomp}
\usepackage{xcolor}
\usepackage{txfonts}
\usepackage{listings}
\usepackage{enumitem}
\usepackage{mathtools}
\usepackage{gensymb}
\usepackage[breaklinks=true]{hyperref}
\usepackage{tkz-euclide} % loads  TikZ and tkz-base
\usepackage{listings}
\usepackage{circuitikz}
\usepackage{graphicx}

%\newcounter{MYtempeqncnt}
\DeclareMathOperator*{\Res}{Res}
%\renewcommand{\baselinestretch}{2}
\renewcommand\thesection{\arabic{section}}
\renewcommand\thesubsection{\thesection.\arabic{subsection}}
\renewcommand\thesubsubsection{\thesubsection.\arabic{subsubsection}}

\renewcommand\thesectiondis{\arabic{section}}
\renewcommand\thesubsectiondis{\thesectiondis.\arabic{subsection}}
\renewcommand\thesubsubsectiondis{\thesubsectiondis.\arabic{subsubsection}}

% correct bad hyphenation here
\hyphenation{op-tical net-works semi-conduc-tor}
\def\inputGnumericTable{}                                 %%

\lstset{
	frame=single,
	breaklines=true,
	columns=fullflexible
}



\newtheorem{theorem}{Theorem}[section]
\newtheorem{problem}{Problem}
\newtheorem{proposition}{Proposition}[section]
\newtheorem{lemma}{Lemma}[section]
\newtheorem{corollary}[theorem]{Corollary}
\newtheorem{example}{Example}[section]
\newtheorem{definition}[problem]{Definition}
\newcommand{\BEQA}{\begin{eqnarray}}
	\newcommand{\EEQA}{\end{eqnarray}}
\newcommand{\define}{\stackrel{\triangle}{=}}
\newcommand\figref{Fig.~\ref}
\newcommand\tabref{Table~\ref}
\bibliographystyle{IEEEtran}
%\bibliographystyle{ieeetr}


\providecommand{\mbf}{\mathbf}
\providecommand{\pr}[1]{\ensuremath{\Pr\left(#1\right)}}
\providecommand{\qfunc}[1]{\ensuremath{Q\left(#1\right)}}
\providecommand{\sbrak}[1]{\ensuremath{{}\left[#1\right]}}
\providecommand{\lsbrak}[1]{\ensuremath{{}\left[#1\right.}}
\providecommand{\rsbrak}[1]{\ensuremath{{}\left.#1\right]}}
\providecommand{\brak}[1]{\ensuremath{\left(#1\right)}}
\providecommand{\lbrak}[1]{\ensuremath{\left(#1\right.}}
\providecommand{\rbrak}[1]{\ensuremath{\left.#1\right)}}
\providecommand{\cbrak}[1]{\ensuremath{\left\{#1\right\}}}
\providecommand{\lcbrak}[1]{\ensuremath{\left\{#1\right.}}
\providecommand{\rcbrak}[1]{\ensuremath{\left.#1\right\}}}
\theoremstyle{remark}
\newtheorem{rem}{Remark}
\newcommand{\sgn}{\mathop{\mathrm{sgn}}}
\providecommand{\abs}[1]{\left\vert#1\right\vert}
\providecommand{\res}[1]{\Res\displaylimits_{#1}}
\providecommand{\norm}[1]{\left\lVert#1\right\rVert}
%\providecommand{\norm}[1]{\lVert#1\rVert}
\providecommand{\mtx}[1]{\mathbf{#1}}
\providecommand{\mean}[1]{E\left[ #1 \right]}
\providecommand{\fourier}{\overset{\mathcal{F}}{ \rightleftharpoons}}
%\providecommand{\hilbert}{\overset{\mathcal{H}}{ \rightleftharpoons}}
\providecommand{\system}{\overset{\mathcal{H}}{ \longleftrightarrow}}
%\newcommand{\solution}[2]{\textbf{Solution:}{#1}}
\newcommand{\solution}{\noindent \textbf{Solution: }}
\newcommand{\cosec}{\,\text{cosec}\,}
\providecommand{\dec}[2]{\ensuremath{\overset{#1}{\underset{#2}{\gtrless}}}}
\newcommand{\myvec}[1]{\ensuremath{\begin{pmatrix}#1\end{pmatrix}}}
\newcommand{\mydet}[1]{\ensuremath{\begin{vmatrix}#1\end{vmatrix}}}
\renewcommand{\abstractname}{Question}

\let\vec\mathbf

	
	\vspace{3cm}
	
	


\newcommand{\permcomb}[4][0mu]{{{}^{#3}\mkern#1#2_{#4}}}
\newcommand{\comb}[1][-1mu]{\permcomb[#1]{C}}

%\IEEEpeerreviewmaketitle

\newcommand \tab [1][1cm]{\hspace*{#1}}
%\newcommand{\Var}{$\sigma ^2$}
\usepackage{amssymb}
\usepackage{amsmath}
\title{
	
\title{GATE 2023 BM 33}
\author{EE23BTECH11213 - MUTHYALA NIKHITHA SRI
}


}
\begin{document}

\maketitle

\textbf{Question:} 
A continuous time, band-limited signal $x\brak{t}$ has its Fourier transform described by:\\
\begin{equation}
 X\brak{f} = \begin{cases} 
1 - \frac{\abs{f}}{200} & \text{if } \abs{f} \leq 200 \\
0 & \text{if } \abs{f} > 200 
\end{cases}  \\
\end{equation}
The signal is uniformly sampled at a sampling rate of 600 Hz. The Fourier transform of the signal is $X_s\brak{f}$. What is the value of $\frac{X_s\brak{600}}{X_s\brak{500}}$? \\

\textbf{Solution: }
\fi

\begin{table}[h]
 	\centering
 	\resizebox{6 cm}{!}{
 		 \begin{tabular}{|c|c|c|}
        \hline
        \textbf{Parameter} & \textbf{Description} & \textbf{Value} \\
        \hline
        $X\brak{f}$ & Fourier transform of $x\brak{t}$ & $\begin{cases} 1 - \frac{\abs{f}}{200} & \text{if } \abs{f} \leq 200 \\ 0 & \text{if } \abs{f} > 200 \end{cases} $  \\
        \hline
        $X_s\brak{f}$ & Fourier transform of sampled signal & ?\\
        \hline
        
    \end{tabular}


 	}
 	\caption{Input Parameters}
    \label{tab:tablenik_33}
 \end{table}

\begin{align} 
X_s\brak{f} &= \frac{1}{600} \sum_{k=-\infty}^{\infty} X\brak{f - 600k} \\
\implies X_s\brak{f+600} &= \frac{X\brak{f}}{600} \end{align}  
\begin{align} 
X_s\brak{600} &= \frac{X\brak{0}}{600} \\
\implies X_s\brak{600} &= \frac{1}{600} \\
X_s\brak{500} &= \frac{X\brak{-100}}{600} \\
\implies X_s\brak{500} &= \frac{1}{2\cdot 600} \\
\frac{X_s\brak{600}}{X_s\brak{500}} &= 2 
\end{align}

\begin{figure}[h!]
    \centering
    \includegraphics[width=\columnwidth]{2023/BM/33/figs/f1.png}
    \caption{Plot of $X\brak{f}$}
    \label{fig:nikh1}
\end{figure}

\begin{figure}[h!]
    \centering
    \includegraphics[width=\columnwidth]{2023/BM/33/figs/f2.png}
    \caption{Plot of $X_s\brak{f}$}
    \label{fig:nikh2}
\end{figure}


%\end{document}

\pagebreak

 \item The magnitude and phase plots of an LTI systems are shown in figure. Find the transfer function.
\begin{figure}[!h]
    \centering
    \includegraphics[width=\columnwidth]{2023/EE/36/figs/gate.png}
    \caption{}
    \label{fig:EEgatefig36.23}
\end{figure}
\begin{enumerate}
    \item $2.511 e^{-0.0032s}$\\
    \item $\frac{e^{-2.514s}}{s+1}$\\
    \item $1.04e^{-2.514s}$\\
    \item $2.511 e^{-1.047s}$\\
\end{enumerate} \hfill{(GATE EE 23)}\\

\solution
\iffalse
\documentclass[journal,12pt,twocolumn]{IEEEtran}
\usepackage{cite}
\usepackage{amsmath,amssymb,amsfonts,amsthm}
\usepackage{algorithmic}
\usepackage{graphicx}
\usepackage{textcomp}
\usepackage{xcolor}
\usepackage{txfonts}
\usepackage{listings}
\usepackage{enumitem}
\usepackage{mathtools}
\usepackage{gensymb}
\usepackage{comment}
\usepackage[breaklinks=true]{hyperref}
\usepackage{tkz-euclide}
\usepackage{listings}
\usepackage{gvv}
\def\inputGnumericTable{}
\usepackage[latin1]{inputenc}
\usepackage{color}
\usepackage{array}
\usepackage{longtable}
\usepackage{calc}
\usepackage{multirow}
\usepackage{hhline}
\usepackage{ifthen}
\usepackage{lscape}

\newtheorem{theorem}{Theorem}[section]
\newtheorem{problem}{Problem}
\newtheorem{proposition}{Proposition}[section]
\newtheorem{lemma}{Lemma}[section]
\newtheorem{corollary}[theorem]{Corollary}
\newtheorem{example}{Example}[section]
\newtheorem{definition}[problem]{Definition}
\newcommand{\BEQA}{\begin{eqnarray}}
    \newcommand{\EEQA}{\end{eqnarray}}
\newcommand{\define}{\stackrel{\triangle}{=}}
\theoremstyle{remark}
\newtheorem{rem}{Remark}
\begin{document}
    
    \bibliographystyle{IEEEtran}
    \vspace{3cm}
    
    \title{Gate 2023 EE Q36}
    \author{EE23BTECH11212 - Manugunta Meghana Sai$^{*}$% <-this % stops a space
    }
    \maketitle
    \newpage
    \bigskip
    
    \renewcommand{\thefigure}{\theenumi}
    \renewcommand{\thetable}{\theenumi}
    
    \vspace{3cm}
    \textbf{Gate 2023 EE Q36} 
    The magnitude and phase plots of an LTI systems are shown in figure. Find the transfer function.\\
    \begin{figure}[h!]
        \centering
        \includegraphics[width=\columnwidth]{figs/gate.jpeg}
        \caption{Graphs}
        \label{fig:1ee36}
    \end{figure}
    \begin{enumerate}
        \item $2.511 e^{-0.0032s}$\\
        \item $\frac{e^{-2.514s}}{s+1}$\\
        \item $1.04e^{-2.514s}$\\
        \item $2.511 e^{-1.047s}$\\
    \end{enumerate}
    
    \solution
    \fi
    From~\figref{fig:EEgatefig36.23}
    \begin{align}
        \abs{{H}\brak{j\omega}}&= 8 \\
        \angle H\brak{j\omega} &= \frac{-\pi}{3} \omega
    \end{align}
    Substituting the values from~\figref{fig:EEgatefig36.23}, magnitude of transfer function is:
    \begin{align}
        8 &= 20\log_{10}(\abs{{H}\brak{j\omega}})\\
        \abs{{H}\brak{j\omega}} &= 10^{0.4} = 2.511
    \end{align}
    Substituting the values from~\figref{fig:EEgatefig36.23}, The direction of the transfer function is:
    \begin{align}
        \frac{H\brak{j\omega}}{\abs{{H}\brak{j\omega}}} = e^{-j\frac{\pi}{3}\omega}
    \end{align}
    \begin{align}
        H\brak{j\omega} &= 2.511 e^{-j\frac{\pi}{3}\omega} \\
        &= 2.511 e^{-1.047s}
    \end{align}
    \setcounter{figure}{1} 
%\end{document}


\newpage
\item The value of the convolution of $f(x) = 3\cos(2x)$ and $g(x) = \frac{1}{3}\sin(2x)$ where $x \in [0, 2\pi)$, at $x = \frac{\pi}{3}$, is (Rounded off to 2 decimal places)\\
\hfill (GATE 2023 GE)\\
\solution
\pagebreak
\item A system is described by the following differential equation
    \[
    0.01 \frac{d^2y(t)}{dt^2} + 0.2\frac{dy(t)}{dt} + y(t) = 6x(t)
    \]
    where time \( t \) is in seconds. If \( x(t) \) is the unit step input applied at \( t = 0 \) s to this system, the magnitude of the output at \( t = 1 \) s is \(\underline{\hspace{2cm}}\). (Round off the answer to two decimal places.)
    \hfill (GATE-2023.BM)\\
    \solution
    \pagebreak

\item In the differential equation $\frac{dy}{dx} + \alpha x y = 0, \alpha$ is a positive constant. If $y = 1.0$ at
$x = 0.0$, and $y = 0.8$ at $x = 1.0$, the value of $\alpha$ is (rounded off to three decimal places).  \hfill(GATE CE 30 2023)\\
\solution
\iffalse
\let\negmedspace\undefined
\let\negthickspace\undefined
\documentclass[journal,12pt,twocolumn]{IEEEtran}
\usepackage{cite}
\usepackage{amsmath,amssymb,amsfonts,amsthm}
\usepackage{algorithmic}
\usepackage{graphicx}
\usepackage{textcomp}
\usepackage{xcolor}
\usepackage{txfonts}
\usepackage{listings}
\usepackage{enumitem}
\usepackage{mathtools}
\usepackage{float}
\usepackage{gensymb}
\usepackage{comment}
\usepackage[breaklinks=true]{hyperref}
\usepackage{tkz-euclide} 
\usepackage{listings}
\usepackage{gvv}                                        
\def\inputGnumericTable{}                                 
\usepackage[latin1]{inputenc}                                
\usepackage{color}                                            
\usepackage{array}          
\usetikzlibrary{positioning, arrows.meta}
\usepackage{longtable}                                       
\usepackage{calc}                                             
\usepackage{multirow}                                         
\usepackage{hhline}                                           
\usepackage{ifthen}                                           
\usepackage{lscape}
\usepackage{amsmath}
\newtheorem{theorem}{Theorem}[section]
\newtheorem{problem}{Problem}
\newtheorem{proposition}{Proposition}[section]
\newtheorem{lemma}{Lemma}[section]
\newtheorem{corollary}[theorem]{Corollary}
\newtheorem{example}{Example}[section]
\newtheorem{definition}[problem]{Definition}
\newcommand{\BEQA}{\begin{eqnarray}}
\newcommand{\EEQA}{\end{eqnarray}}
\newcommand{\define}{\stackrel{\triangle}{=}}
\theoremstyle{remark}
\newtheorem{rem}{Remark}
\begin{document}

\bibliographystyle{IEEEtran}
\title{GATE-EE-Q14}
\author{EE23BTECH11015 - DHANUSH V NAYAK$^{*}$% <-this % stops a space
}
\maketitle
\newpage
\bigskip
\renewcommand{\thefigure}{\arabic{figure}}
\renewcommand{\thetable}{\theenumi}
\textbf{Question:}Consider a unity-gain negative feedback system consisting of the plant $G\brak{s}$  and a proportional-integral controller. Let the proportional gain and integral
gain be 3 and 1, respectively. For a unit step reference input, the final values of the
controller output and the plant output, respectively, are
\begin{align}
    G\brak{s} = \frac{1}{\brak{s-1}} \notag
\end{align}
\solution 
\fi
\begin{table}[H]
\centering
\renewcommand\thetable{1}
\setlength{\extrarowheight}{9pt}
\resizebox{0.5\textwidth}{!}{
\begin{tabular}{|c|c|c|}
\hline
\textbf{Parameter} & \textbf{Description} & \textbf{Value} \\ \hline
$K_{p}$ & Proportional Gain & 3  \\ \hline
$K_{i}$ & Integral Gain &1 \\ \hline
$r\brak{t}$& Reference Input & $u\brak{t}$ \\ \hline 
$w\brak{t}$& Controller Output & $?$ \\ \hline 
$y\brak{t}$ & Plant Output & $?$ \\ \hline
$e\brak{t}$ & Error Input & $r\brak{t}-y\brak{t}$ \\ \hline
\end{tabular}}
\caption{Parameter Table}
\label{tab:gate_ee_Q14}
\end{table}

From the\figref{fig:gate_ee_Q14_blockdiagram}:
\begin{align}
    E\brak{s}&= U\brak{s} - Y\brak{s}\label{eq:gate_ee_Q14.1}\\
W\brak{s} &= 3E\brak{s} + \frac{1}{s}E\brak{s}\label{eq:gate_ee_Q14.2}\\
    Y\brak{s} &= G\brak{s}W\brak{s} \label{eq:gate_ee_Q14.3}
\end{align}
Some results:
\begin{align}
    tx\brak{t} &\system{L} -\frac{d{X\brak{s}}}{ds} \label{eq:laplace_diff_prop}\\
    e^{-at}x\brak{t} &\system{L} X\brak{s+a}\label{eq:laplace_timeshifting_prop}
\end{align}
By using \eqref{eq:laplace_diff_prop} and \eqref{eq:laplace_timeshifting_prop}:
\begin{align}
    e^{-t}u\brak{t} &\system{L} \frac{1}{s+1} ,  Re\brak{s}>-1 \label{eq:gate_ee_Q14result.1}\\
    t e^{-t}u\brak{t} &\system{L} \frac{1}{\brak{s+1}^2},  Re\brak{s}>-1 
 \label{eq:gate_ee_Q14result.2}
\end{align}

\begin{figure}[H]
    \resizebox{0.9\textwidth}{!}{\tikzset{
    block/.style = {draw, fill=white, rectangle, minimum height=3em, minimum width=3em},
    tmp/.style  = {coordinate}, 
    minus/.style= {draw, fill=white, circle, node distance=1cm, append after command={\pgfextra \draw ($(\tikzlastnode.center) + (-0.15,0)$) -- ($(\tikzlastnode.center) + (0.15,0)$) node[above] {$-$}; \endpgfextra}},
    plus/.style= {draw, fill=white, circle, node distance=1cm, append after command={\pgfextra \draw ($(\tikzlastnode.center) + (-0.15,0)$) -- ($(\tikzlastnode.center) + (0.15,0)$) node[above] {$+$}; \endpgfextra}},
    input/.style = {coordinate},
    output/.style= {coordinate},
    pinstyle/.style = {pin edge={to-,thin,black}}
}


\begin{tikzpicture}[auto, node distance=2cm,>=latex]
    \node [input, name=rinput] (rinput) {};
    \node [minus, right of=rinput] (sum1) {};
    
    \node [block, right of=sum1] (controller) {$k_{p}=3$};
    \node [block, above of=controller, node distance=2cm] (up) {$\frac{k_{i}}{s}=\frac{3}{s}$};
    
    \node [plus, right of=controller, node distance=2cm] (sum2) {};
    \node [block, right of=sum2, node distance=3.5cm] (system) {$G\brak{s}=\frac{1}{\brak{s-1}}$};
    \node [output, right of=system, node distance=2cm] (output) {};
    \node [tmp, below of=controller] (tmp1) {$H(s)$};

    \draw [->] (rinput) -- node[below]{$r\brak{t}$} (sum1);
    \draw [->] (sum1) -- node[name=z,anchor=north,fill=white,circle,inner sep=1pt]{$e\brak{t}$} (controller);
    \draw [->] (controller) -- (sum2);
    \draw [->] (sum2) -- node[above, pos=0.8]{$w\brak{t}$} (system);
    \draw [->] (system) -- node [name=y] {$y\brak{t}$} (output);
    \draw [->] (z) |- (up);
    \draw [->] (up) -| (sum2);
    \draw [->] (y) |- (tmp1) -| (sum1);
\end{tikzpicture}
}
    \caption{Block Diagram of System}
    \label{fig:gate_ee_Q14_blockdiagram}
\end{figure}
\begin{enumerate}
\item \textbf{Plant Output:}\\
From \eqref{eq:gate_ee_Q14.1} , \eqref{eq:gate_ee_Q14.2} and \eqref{eq:gate_ee_Q14.3}:
\begin{align}
    Y\brak{s} &=  \frac{3s+1}{s\brak{s+1}^2} ,  Re\brak{s}>-1 \label{eq:Y(s)}
\end{align}
Final Value Theorem:    
\begin{align}
    \lim_{t \to \infty} x\brak{t}&= \lim_{s \to 0} sX\brak{s}\label{eq:finalval_thm}
\end{align}
Using \eqref{eq:finalval_thm} on Y\brak{s}:
\begin{align}
     \lim_{t \to \infty} y\brak{t}&= \lim_{s \to 0} sY\brak{s}\\
                            &= 1
\end{align}

Taking partial fraction of \eqref{eq:Y(s)} :
\begin{align}
    Y\brak{s} &= \frac{1}{s} + \frac{2}{\brak{s+1}^2} - \frac{1}{s+1}
\end{align}
Using \eqref{eq:gate_ee_Q14result.1} and \eqref{eq:gate_ee_Q14result.2}:
\begin{align}
    \therefore y\brak{t} &= u\brak{t}+ 2t e^{-t}u\brak{t} - e^{-t}u\brak{t}
\end{align}
\item \textbf{Controller Output:}\\
From \eqref{eq:gate_ee_Q14.2}
\begin{align}
     W\brak{s} &= \frac{3}{s} + \frac{1}{s^2} - Y\brak{s}\brak{3+\frac{1}{s}}
\end{align}
Substituting \eqref{eq:Y(s)}
\begin{align}
    W\brak{s} &= \frac{\brak{s-1}\brak{3s+1}}{s\brak{s+1}^2} ,  Re\brak{s}>-1 \label{eq:W(s)}
\end{align}
Using \eqref{eq:finalval_thm} on W\brak{s}
\begin{align}
     \lim_{t \to \infty} w\brak{t}&= \lim_{s \to 0} sW\brak{s}\\
                            &= -1
\end{align}
Taking partial fraction of equation\eqref{eq:W(s)} :
\begin{align}
    W\brak{s} &= -\frac{1}{s} - \frac{4}{\brak{s+1}^2} + \frac{4}{s+1}
\end{align}
Using equations \eqref{eq:gate_ee_Q14result.1} and \eqref{eq:gate_ee_Q14result.2} and taking inverse lapalace transform:
\begin{align}
    w\brak{t} &= -u\brak{t}-4t e^{-t}u\brak{t} +4 e^{-t}u\brak{t}
\end{align}

\begin{figure}[H]
    \includegraphics[width=1\columnwidth]{2023/EE/14/figs/Plot of w(t).png}
    \caption{$w\brak{t}$ converges at -1.}
    \label{fig:w_t}
\end{figure}

\begin{figure}[H]
    \includegraphics[width=1\columnwidth]{2023/EE/14/figs/Plot of y(t).png}
    \caption{$y\brak{t}$ converges at +1}
    \label{fig:y_t}
\end{figure}

\end{enumerate}
%\end{document}


\pagebreak

\end{enumerate}
