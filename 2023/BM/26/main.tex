\iffalse
\let\negmedspace\undefined
\let\negthickspace\undefined
\documentclass[journal,12pt,twocolumn]{IEEEtran}
\usepackage{cite}
\usepackage{amsmath,amssymb,amsfonts}
\usepackage{graphicx}
\usepackage{textcomp}
\usepackage{xcolor}
\usepackage{txfonts}
\usepackage{listings}
\usepackage{enumitem}
\usepackage{mathtools}
\usepackage{gensymb}
\usepackage{comment}
\usepackage[breaklinks=true]{hyperref}
\usepackage{tkz-euclide} 
\usepackage{listings}
\usepackage{gvv}                                        
\def\inputGnumericTable{}                                 
\usepackage[latin1]{inputenc}                                
\usepackage{color}                                            
\usepackage{array}                                            
\usepackage{longtable}                                       
\usepackage{calc}                                             
\usepackage{multirow}                                         
\usepackage{hhline}                                           
\usepackage{ifthen}                                           
\usepackage{lscape}
\usepackage[export]{adjustbox}

\newtheorem{theorem}{Theorem}[section]
\newtheorem{problem}{Problem}
\newtheorem{proposition}{Proposition}[section]
\newtheorem{lemma}{Lemma}[section]
\newtheorem{corollary}[theorem]{Corollary}
\newtheorem{example}{Example}[section]
\newtheorem{definition}[problem]{Definition}
\newcommand{\BEQA}{\begin{eqnarray}}
\newcommand{\EEQA}{\end{eqnarray}}
\newcommand{\define}{\stackrel{\triangle}{=}}
\newtheorem{rem}{Remark}

\begin{document}
\parindent 0px
\bibliographystyle{IEEEtran}

\vspace{3cm}

\title{}
\author{EE23BTECH11042 -  Khusinadha Naik$^{*}$
}
\maketitle
\newpage
\bigskip

% \renewcommand{\thefigure}{\theenumi}
% \renewcommand{\thetable}{\theenumi}


\noindent \textbf{26.} \hspace{2pt}A casual, discrete time system is described by the difference equation $y[n] = 0.5 y[n-1] + x[n]$, for all $n$, where $y[n]$ denotes the output sequence and $x[n]$ denotes the input sequence. Which of the following statements is/are TRUE?
\begin{flushright}
\hfill(GATE 2023 BM)
\end{flushright}
\begin{enumerate}[label = (\alph*)]
	\item The system has an impulse response described by $0.5^{n} u[-n]$ where $u[n]$ is the  
unit step sequence. 		
	\item The system is stable in the bounded input, bounded output sense.		
	\item The system has an infinite number of non-zero samples in its impulse response	
	\item The system has a finite number of non-zero samples in its impulse response.
\end{enumerate}

\noindent \textbf{Ans.}\\
\fi
\begin{table}[h]
\centering
\begin{tabular}{|c|c|c|}
        \hline
        \textbf{Parameter} & \textbf{Value} & \textbf{Description} \\
        \hline
        $x[n]$ & ? & Input Sequence \\
        \hline
        $y[n]$ & ? & Output Sequence \\
        \hline
\end{tabular}
\caption{Input parameters table}
\label{tab:GATE.2023.BM.26.1}





\end{table}
\begin{align}
y[n] = 0.5y[n-1] + x[n] 
\end{align}

Taking $Z$-Transform 
\begin{align}
Y\brak{z} &= 0.5z^{-1}Y\brak{z} + X\brak{z} \\
\implies \frac{Y\brak{z}}{X\brak{z}} &= \frac{1}{1 - 0.5z^{-1}} = H\brak{z} 
\end{align}
If $x[n]$ is impulse input 
\begin{align}
\implies &Y\brak{z} = H\brak{z} = \frac{1}{1 - 0.5z^{-1}}  \label{eq:GATE.2023.BM.26.4}
\end{align}
From \eqref{eq:GATE.2023.BM.26.4} pole lies at $z = 0.5$
\begin{align}
a^{n}u\brak{n} \xleftrightarrow{\mathcal{Z}} &\frac{1}{1 - az^{-1}} \quad , \abs{z} > a \label{eq:GATE.2023.BM.26.5}
\end{align}

From \eqref{eq:GATE.2023.BM.26.4} , \eqref{eq:GATE.2023.BM.26.5}
\begin{align}
h[n] = 0.5^{n}u[n] \quad , \abs{z} > 0.5 \label{eq:GATE.2023.BM.26.6}
\end{align}
\pagebreak 
Plotting $h[n]$ vs $n$
\begin{figure}[h]
    \includegraphics[width=0.5\textwidth]{2023/BM/26/figs/fig1.png}
    \caption{Plot of $h[n]$ vs $n$}
    \label{fig:GATE.2023.BM.26.1}
\end{figure}

\begin{enumerate}
\item From \eqref{eq:GATE.2023.BM.26.6} , (a) is wrong
\item As pole lies within unit circle (b) is true
\item From \eqref{eq:GATE.2023.BM.26.6} and \figref{fig:GATE.2023.BM.26.1} ,(c) is true and hence
\item (d) is false 
\end{enumerate}





%\end{document}
