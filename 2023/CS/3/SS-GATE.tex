\iffalse
\let\negmedspace\undefined
\let\negthickspace\undefined
\documentclass[journal,12pt,onecolumn]{IEEEtran}
\usepackage{cite}
\usepackage{amsmath,amssymb,amsfonts}
\usepackage{graphicx}
\usepackage{textcomp}
\usepackage{xcolor}
\usepackage{txfonts}
\usepackage{listings}
\usepackage{enumitem}
\usepackage{mathtools}
\usepackage{gensymb}
\usepackage{comment}
\usepackage[breaklinks=true]{hyperref}
\usepackage{tkz-euclide} 
\usepackage{listings}
\usepackage{gvv}                                        
\def\inputGnumericTable{}                                 
\usepackage[latin1]{inputenc}                                
\usepackage{color}                                            
\usepackage{array}                                            
\usepackage{longtable}                                       
\usepackage{calc}                                             
\usepackage{multirow}                                         
\usepackage{hhline}                                           
\usepackage{ifthen}                                           
\usepackage{lscape}
\usepackage[export]{adjustbox}
\newtheorem{theorem}{Theorem}[section]
\newtheorem{problem}{Problem}
\newtheorem{proposition}{Proposition}[section]
\newtheorem{lemma}{Lemma}[section]
\newtheorem{corollary}[theorem]{Corollary}
\newtheorem{example}{Example}[section]
\newtheorem{definition}[problem]{Definition}
\newcommand{\BEQA}{\begin{eqnarray}}
	\newcommand{\EEQA}{\end{eqnarray}}
\newcommand{\define}{\stackrel{\triangle}{=}}
\newtheorem{rem}{Remark}

\begin{document}
	\parindent 0px
	\bibliographystyle{IEEEtran}
	

	
	\title{}
	\author{EE23BTECH11209 - K S Ballvardhan$^{*}$
	}
	\maketitle
	\bigskip
	
	% \renewcommand{\thefigure}{\theenumi}
	% \renewcommand{\thetable}{\theenumi}
	
	
	
	
	\textbf{Question:} A series of natural numbers F$_1$, F$_2$, F$_3$, F$_4$, F$_5$, F$_6$, F$_7$,....obeys F$_{n+1}$ = F$_n$ + F$_{n-1}$ for all integers n $\geq$ 2.
	If F$_6$ = 37, and F$_7$ = 60, then what is F$_1$? \hfill[GATE CS 2023]
	
	\solution
        \fi
	
	\begin{table}[ht] 
		\centering
		\begin{tabular}{|c|c|c|}
    \hline
    \textbf{Parameter} & \textbf{Description} & \textbf{Value} \\
    \hline
    $ x\brak 6$ & Seventh term of the sequence & 60 \\
    \hline
    $ x\brak 5$ & Sixth term of the sequence & 37 \\
    \hline
    $ x\brak 1$ & Second term of the sequence & ? \\
    \hline
    $ x\brak 0$ & First term of the sequence & ? \\
    \hline
    \end{tabular}


		\caption{input values}
		\label{tab: Table2023cs3}
	\end{table}
	
    Taking z-transform of $X\brak z$:
	\begin{align}
		X\brak{z} &= x\brak 0 + z^{-1} x\brak 1 + \sum_{n=2}^{\infty} x\brak n z^{-n}\\
	    &= x\brak 0 + z^{-1} x\brak 1 + z^{-1} \sum_{n=1}^{\infty} x\brak {n+1} z^{-n}\\
	    &= x\brak 0 + z^{-1} x\brak 1 + z^{-1} \sum_{n=1}^{\infty} \brak{x\brak n + x\brak {n-1}}z^{-n} \\
	    &= x\brak 0 + z^{-1} x\brak 1 + z^{-1}\brak{X\brak{z}-x\brak 1 + z^{-1}X\brak{z}} \\
		\implies X\brak{z}&= \frac{x\brak 0 + \brak{x\brak 1-x\brak 0}z^{-1}}{1-z^{-1}-z^{-2}}
	\end{align}
	Using Contour Integration to find the inverse $Z$-transform,
	\begin{align}
		x\brak n &=\frac{1}{2\pi j}\oint_{C} X\brak{z} \;z^{n-1} \;dz  \\
		&=\frac{1}{2\pi j}\oint_{C}\frac{z^{n}\brak{x\brak 1-x\brak 0+x\brak 0 z}}{z^2-z-1} \;dz 
	\end{align}
	
	By residue theorem:
	
	\begin{align}
		x\brak n &=\frac{1}{\brak {0}!}\lim\limits_{z\to {\frac{1+\sqrt{5}}{2}}}\frac{d}{dz}\brak {{(z+\frac{1+\sqrt{5}}{2})}X\brak{z}} + \frac{1}{\brak {0}!}\lim\limits_{z\to \frac{1-\sqrt{5}}{2}}\frac{d}{dz}\brak {{\brak{z+\frac{1-\sqrt{5}}{2}}}X\brak{z}}
	\end{align}
	On simplifying we get,
	\begin{align}
		x\brak n &= \brak{x\brak1-x\brak0}\brak{\brak{\frac{1+\sqrt{5}}{2}}^n-\brak{\frac{1-\sqrt{5}}{2}}^n} + \brak{x\brak 0}\brak{\brak{\frac{1+\sqrt{5}}{2}}^{n+1}-\brak{\frac{1-\sqrt{5}}{2}}^{n+1}}
	\end{align}
	From the values in \tabref{tab: Table2023cs3}:
	\begin{align}
		5 \brak{x\brak 1-x\brak 0} + 8 x\brak 0 &=37\\
		8 \brak{x\brak 1-x\brak 0} + 13 x\brak 0 &=60\\
		\implies x\brak 1 =5, x\brak 0 &=4
	\end{align}
	\begin{align}
		\therefore {x\brak 0=4}
	\end{align}
	\begin{figure}[ht]
		\includegraphics[width = \columnwidth]{2023/CS/3/figs/fig4.png}
		\caption{Terms of the given sequence}
		\centering
		\label{fig: fig4}
	\end{figure}
%\end{document}
