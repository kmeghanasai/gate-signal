\begin{enumerate}[label=\thechapter.\arabic*,ref=\thechapter.\theenumi]
\item Let y\brak{t}=x\brak{4t},where x\brak{t} is a continous-time periodic signal of $100$s.the fundamental period of y\brak{t} is (\textbf{rounded off to the nearest integer})
 \hfill(GATE IN 2023)\\
\solution
\iffalse
\let\negmedspace\undefined
\let\negthickspace\undefined
\documentclass[journal,12pt,twocolumn]{IEEEtran}
\usepackage{cite}
\usepackage{amsmath,amssymb,amsfonts,amsthm}
\usepackage{algorithmic}
\usepackage{graphicx}
\usepackage{textcomp}
\usepackage{xcolor}
\usepackage{txfonts}
\usepackage{listings}
\usepackage{enumitem}
\usepackage{mathtools}
\usepackage{gensymb}
\usepackage{comment}
\usepackage[breaklinks=true]{hyperref}
\usepackage{tkz-euclide} 
\usepackage{listings}
\usepackage{gvv}                                        
\def\inputGnumericTable{}                                 
\usepackage[latin1]{inputenc}                                
\usepackage{color}                                            
\usepackage{array}                                            
\usepackage{longtable}                                       
\usepackage{calc}                                             
\usepackage{multirow}                                         
\usepackage{hhline}                                           
\usepackage{ifthen}                                           
\usepackage{lscape}

\newtheorem{theorem}{Theorem}[section]
\newtheorem{problem}{Problem}
\newtheorem{proposition}{Proposition}[section]
\newtheorem{lemma}{Lemma}[section]
\newtheorem{corollary}[theorem]{Corollary}
\newtheorem{example}{Example}[section]
\newtheorem{definition}[problem]{Definition}
\newcommand{\BEQA}{\begin{eqnarray}}
\newcommand{\EEQA}{\end{eqnarray}}
\newcommand{\define}{\stackrel{\triangle}{=}}
\theoremstyle{remark}
\newtheorem{rem}{Remark}
\begin{document}

\bibliographystyle{IEEEtran}
\vspace{3cm}

\title{GATE 2023 IN 29}
\author{EE23BTECH11065 - prem sagar}
\maketitle
\newpage

\bigskip 

\renewcommand{\thefigure}{\theenumi}
\renewcommand{\thetable}{\theenumi}
\textbf{Question}:
\\\\Let y\brak{t}=x\brak{4t},where x\brak{t} is a continous-time periodic signal of $100$s.the fundamental period of y\brak{t} is (\textbf{rounded off to the nearest integer})
 \hfill(GATE IN 29)
 \\\\\textbf{Solution}:
\fi
\begin{table}[!ht]
\def\arraystretch{1.5}
   \centering
    \renewcommand\thetable{1}
      \begin{tabular}{|c|c|c|}
   \hline
   \textbf{Symbol} & \textbf{Value}& \textbf{Description} \\
   \hline
         $T$ & $100$ & fundamental period of $x\brak{t}$\\
        \hline
        $T_1$ &  & fundamental period of y\brak{t}\\
        \hline
        $\omega_0$ & $\frac{8\pi}{100} $  & fundamental frequency of y\brak{t}\\
        \hline
\end{tabular}

    \caption{input parameters}
    \label{tab:IN 29}
 \end{table}
\\From \tabref{tab:IN 29}
\\Applying Fourier series:
 \begin{align}
 x\brak{t}&=\sum_{n=-\infty}^{\infty}c_ne^\frac{j\;2\pi n\;t}{100}
\\ y\brak{t}&=x\brak{4t}
\\ y\brak{t}&=\sum_{n=-\infty}^{\infty}c_ne^\frac{j\;2\pi n\;\brak{4t}}{100}
\\&=\sum_{n=-\infty}^{\infty}c_ne^\frac{j\;2\pi n\;t}{25}
\\T_1&=25\text{sec}
 \end{align}
\begin{figure}[h]
 \renewcommand\thefigure{1}
    \centering
    \includegraphics[width=1\linewidth]{2023/IN/29/figs/figr.png}
    \caption{plot y\brak{t} v/s t}
\end{figure}
%\end{document}

\newpage

\item In the circuit shown below, it is observed that the amplitude of voltage across the resistor is the same as the amplitude of the sorce voltage. What is the angular frequency $\omega_0$(in rad$/s$)?\\
\begin{circuitikz}[american]
    \draw (0,0) to[R, l=$10K\Omega$] (2,0) to[L, l=$10mH$] (4,0) to[C, l=$1\mu{F}$] (6,0) -- (6,-1) 
    to[sV, l=$100\cos(\omega_0 t)$] (0,-1) -- (0,0)
    (0,-1) node[circ]{} node[left]{$+$}
    (6,-1) node[circ]{} node[right]{$-$};
\end{circuitikz}
\begin{enumerate}
    \item[(A)] $10^4$\\
    \item[(B)] $10^3$\\
    \item[(C)] $10^3\pi$\\
    \item[(D)] $10^4\pi$  
\end{enumerate} \hfill(GATE BM 2023)\\
\solution
% \iffalse
\let\negmedspace\undefined
\let\negthickspace\undefined
\documentclass[journal,12pt,twocolumn]{IEEEtran}
\usepackage{cite}
\usepackage{amsmath,amssymb,amsfonts,amsthm}
\usepackage{algorithmic}
\usepackage{graphicx}
\usepackage{textcomp}
\usepackage{xcolor}
\usepackage{txfonts}
\usepackage{listings}
\usepackage{enumitem}
\usepackage{mathtools}
\usepackage{gensymb}
\usepackage{comment}
\usepackage{tikz}
\usepackage[breaklinks=true]{hyperref}
\usepackage{tkz-euclide} 
\usepackage{listings}
\usepackage{gvv}
\def\inputGnumericTable{}
\usepackage[latin1]{inputenc}                              
\usepackage{color}                                            
\usepackage{array}                                            
\usepackage{longtable}                                       
\usepackage{calc}                                             
\usepackage{multirow}   
\usetikzlibrary{circuits.ee.IEC}
\usepackage{hhline}                                           
\usepackage{ifthen} 
\usepackage{circuitikz}
\usepackage{lscape}
\newtheorem{theorem}{Theorem}[section]
\newtheorem{problem}{Problem}
\newtheorem{proposition}{Proposition}[section]
\newtheorem{lemma}{Lemma}[section]
\newtheorem{corollary}[theorem]{Corollary}
\newtheorem{example}{Example}[section]
\newtheorem{definition}[problem]{Definition}
\newcommand{\BEQA}{\begin{eqnarray}}
\newcommand{\EEQA}{\end{eqnarray}}
\newcommand{\define}{\stackrel{\triangle}{=}}
\theoremstyle{remark}
\newtheorem{rem}{Remark}
\begin{document}

\bibliographystyle{IEEEtran}
\vspace{3cm}

\title{GATE-2023 BM Q-42}
\author{EE23BTECH11207 -KAILASH.C$^{*}$% <-this % stops a space
}
\maketitle
\newpage
\bigskip

\renewcommand{\thefigure}{\theenumi}
\renewcommand{\thetable}{\theenumi}
In the circuit shown below, it is observed that the amplitude of voltage across the resistor is the same as the amplitude of the sorce voltage. What is the angular frequency $\omega_0\brak{in rad/s}$?
\begin{circuitikz}[american]
    \draw (0,0) to[R, l=$10K\Omega$] (2,0) to[L, l=$10mH$] (4,0) to[C, l=$1\mu{F}$] (6,0) -- (6,-1) 
    to[sV, l=$100\cos\brak{\omega_0 t}$] (0,-1) -- (0,0)
    (0,-1) node[circ]{} node[left]{$+$}
    (6,-1) node[circ]{} node[right]{$-$};
\end{circuitikz}

\begin{enumerate}
    \item[(A)] $10^4$\\
    \item[(B)] $10^3$\\
    \item[(C)] $10^3\pi$\\
    \item[(D)] $10^4\pi$  
\end{enumerate}
\solution
\begin{table}[h]
\begin{tabular}{|l|l|l|}
\hline
\textbf{Symbols} & \textbf{Parameters} & \textbf{Value}\\ \hline
R & Resistance & $10K\Omega$ \\ \hline
L & Inductance & $10mH$ \\ \hline
C & Capacitance & $1\mu{F}$\\ \hline
$\omega_0$ & Angular Frequency & \\ \hline
$V_s$ & Source Voltage & \\ \hline
\end{tabular}
\caption{Parameter Table}
\label{tab:gate.bm.42}
\end{table}
\\
\begin{circuitikz}[american]
    \draw (0,0) to[R, l=$10K\Omega$] (2,0) to[L, l=$10^{-2}j\omega_0$] (4,0) to[C, l=$\frac{10^6}{j\omega_0}$] (6,0) -- (6,-1) 
    to[sV, l=$100\cos\brak{\omega_0 t}$] (0,-1) -- (0,0)
    (0,-1) node[circ]{} node[left]{$+$}
    (6,-1) node[circ]{} node[right]{$-$};
\end{circuitikz}

\\
We have:
\begin{align}
    V_R&=V_s \label{eq:421}
\end{align}
Using KVL:
\begin{align}
    V_s&=V_R+V_C+V_L\label{eq:422}
\end{align}
By using \eqref{eq:421} in \eqref{eq:422}:
\begin{align}
    V_C&=-V_L\label{eq:423}\\
    X_C&=-X_L\label{eq:424}\\
    \frac{1}{j\omega_0C}&=-j\omega_0L\label{eq:425}\\
\frac{1}{LC} &= -j^2\omega_0^{2} \label{eq:426}\\ 
\omega_0^{2}&=\frac{1}{LC}\label{eq:427}\\
\omega_0&=\frac{1}{\sqrt{LC}} \label{eq:428}\\
&=\frac{1}{\sqrt{10^{-2}\times10^{-6}}}\label{eq:429}\\
&=\frac{1}{10^{-4}}\label{eq:4210}\\
&=10^4 rad/s\label{eq:4211}
\end{align}
\begin{figure}[h]
        \centering
\includegraphics[width=\columnwidth]{Figure42.png}
    \caption{Voltage across Resistor and Source voltage}
    \label{fig:plot42}
\end{figure}


\end{document}

\newpage

\item For a regular sinusoidal wave propagating in deep water having wave height of 3.5 m and wave period of 9 s, the wave steepness is \underline{\hspace{1cm}} (round off to three decimal places).
\hfill Gate 2023 NM 33\\
\solution
\input{2023/NM/33/g2.tex}
\newpage

\item  A spring mass system is shown in the figure . Take the value of acceleration  due to gravity as $g=9.81m/s^2$.The static deflection due to weight and the time period of the oscillations,respectively,are\\
 \begin{figure}[h!]
    \centering
    \includegraphics[width = \columnwidth]{2023/XE/71/figs/fig1.jpg}
\end{figure}
\hfill{(GATE 2023 XE)}\\
\solution
\iffalse
\let\negmedspace\undefined
\let\negthickspace\undefined
\documentclass[journal,12pt,twocolumn]{IEEEtran}
\usepackage{cite}
\usepackage{amsmath,amssymb,amsfonts,amsthm}
\usepackage{algorithmic}
\usepackage{graphicx}
\usepackage{textcomp}
\usepackage{xcolor}
\usepackage{txfonts}
\usepackage{listings}
\usepackage{enumitem}
\usepackage{mathtools}
\usepackage{gensymb}
\usepackage{comment}
\usepackage[breaklinks=true]{hyperref}
\usepackage{tkz-euclide} 
\usepackage{listings}
\usepackage{gvv}                                        
\def\inputGnumericTable{}                                 
\usepackage[latin1]{inputenc}                                
\usepackage{color}                                            
\usepackage{array}                                            
\usepackage{longtable}                                       
\usepackage{calc}                                             
\usepackage{multirow}                                         
\usepackage{hhline}                                           
\usepackage{ifthen}                                           
\usepackage{lscape}

\newtheorem{theorem}{Theorem}[section]
\newtheorem{problem}{Problem}
\newtheorem{proposition}{Proposition}[section]
\newtheorem{lemma}{Lemma}[section]
\newtheorem{corollary}[theorem]{Corollary}
\newtheorem{example}{Example}[section]
\newtheorem{definition}[problem]{Definition}
\newcommand{\BEQA}{\begin{eqnarray}}
 \newcommand{\EEQA}{\end{eqnarray}}
\newcommand{\define}{\stackrel{\triangle}{=}}
\theoremstyle{remark}
\newtheorem{rem}{Remark}
\begin{document}
 \bibliographystyle{IEEEtran}
 \vspace{3cm}
 \title{\textbf{XE 71}}
 \author{EE23BTECH11048-Ponugumati Venkata Chanakya$^{*}$% <-this % stops a space
 }
 \maketitle
 \newpage
 \bigskip
 \renewcommand{\thefigure}{\theenumi}
 \renewcommand{\thetable}{\theenumi}
 \textbf{QUESTION:}
 A spring mass system is shown in the figure . Take the value of acceleration  due to gravity as $g=9.81m/s^2$.The static deflection due to weight and the time period of the oscillations,respectively,are\\
 \begin{figure}[h!]
    \centering
    \includegraphics[width = \columnwidth]{2023/XE/71/figs/xe_71_f1.png}
\end{figure}
\hfill{(GATE2023 XE)}\\
\solution
\fi
\begin{enumerate}
    \item Static deflection due to weight(sdw)\\
    let x be sdw.\\0
    At mean position in equilibrium\\
    \ref{XE_71.t1}
    \begin{align}
        Mg&=kx\\
        x&=39.24cm
    \end{align}
     \item Time period of oscillation\\
     \begin{align}
           F&=-kx\\
           m\brak{\frac{d^2x}{dt^2}}&=-kx
     \end{align}
      Initial Conditions be at extreme  point of SHM
      \begin{align}
       x(0)&=0.3924 \label{XE_71.5}\\
       \frac{dx}{dt}&=0 \text{ at } t=0 \text{ (released from rest)} \label{XE_71.6}
        \end{align} 
     Taking Laplace transform:
     \begin{align}
     m(s^2X(s)-sx(0)-mx'(0))+kX(s)&=0
     \end{align}
     \begin{align}
      X(s) &= \frac{x(0)ms+mx'(0)}{ms^2 + k}\\
      X(s)&=x(0)\frac{s}{s^2+\frac{k}{m}}+\brak{x'(0)\sqrt{\frac{k}{m}}}\frac{\sqrt{\frac{k}{m}}}{s^2+\brak{\sqrt{\frac{k}{m}}}^2}
      \end{align}
     Taking Inverse Laplace Transform:\\
     \begin{align}
     x(t)&=x(0)\cos\left(\sqrt{\frac{k}{m}}t\right)+\brak{x'(0)\sqrt{\frac{k}{m}}}\sin \brak{\sqrt{\frac{k}{m}}t}\\
     \end{align}
     Using \ref{XE_71.5} and \ref{XE_71.6} 
     \begin{align}
      x(t) &= 0.3924 \cos\left(\sqrt{\frac{k}{m}}t\right)\\
      x(t)&=39.24 \sin\brak{5t+\frac{\pi}{2}} \text{ cm}
     \end{align}
    The static deflection due to weight and the time period of the oscillations,respectively are $39.24$ cm and $\frac{2\pi}{5}$ s
\end{enumerate}
 \begin{figure}[h!]
    \centering
    \includegraphics[width = \columnwidth]{2023/XE/71/figs/xe_71_f2.png}
\end{figure}
 \begin{table}[!ht]
    \centering
             \begin{tabular}{|c|c|c|} 
      \hline
\textbf{Variable}& \textbf{Description}& \textbf{Value}\\\hline
         $M$& weight of block &$4$ kg\\\hline
          $K$ & spring constant & $100\frac{N}{m}$  \\\hline
          $x$& Static deflection due to weight&$39.24$ cm\\\hline  
          $x(t)$& Displacement of particle from mean position at time t  & none \\\hline
          $x(0)$& Initial Displacement of particle from mean position  & $39.24$cm \\\hline
           $x'(t)$& velocity of particle & none \\\hline
          $x'(0)$& initial velocity of particle & $0$ \\\hline
    \end{tabular}

    \caption{input parameters}
     \label{XE_71.t1}
\end{table}
%\end{document}


\pagebreak

\item In the circuit shown below, the amplitudes of the voltage across the resistor and the capacitor are equal. What is the value of the angular frequency $\omega_o$ (in rad/s)? 
(Round off the answer to one decimal place.) \hfill(GATE BM 32 2023)
\begin{circuitikz}
    % Voltage source
    \draw (0,0) to[sV, v=$100\cos(\omega_{o} t)$] (0,2);
    
    % Resistor
    \draw (0,2) to[R, l=$1\text{ k}\Omega$] (3,2);
    
    % Capacitor
    \draw (3,2) to[C, l=$100\mu\text{F}$] (3,0);
    
    % Ground
    \draw (3,0) -- (0,0);
\end{circuitikz}
\solution
\iffalse
\let\negmedspace\undefined
\let\negthickspace\undefined
\documentclass[journal,12pt,twocolumn]{IEEEtran}
\usepackage{cite}
\usepackage{amsmath,amssymb,amsfonts,amsthm}
\usepackage{algorithmic}
\usepackage{graphicx}
\usepackage{textcomp}
\usepackage{xcolor}
\usepackage{txfonts}
\usepackage{listings}
\usepackage{enumitem}
\usepackage{mathtools}
\usepackage{gensymb}
\usepackage[breaklinks=true]{hyperref}
\usepackage{tkz-euclide} % loads  TikZ and tkz-base
\usepackage{listings}
\usepackage{gvv}
\usepackage{circuitikz}

\newtheorem{theorem}{Theorem}[section]
\newtheorem{problem}{Problem}
\newtheorem{proposition}{Proposition}[section]
\newtheorem{lemma}{Lemma}[section]
\newtheorem{corollary}[theorem]{Corollary}
\newtheorem{example}{Example}[section]
\newtheorem{definition}[problem]{Definition}

\newcommand{\BEQA}{\begin{eqnarray}}
\newcommand{\EEQA}{\end{eqnarray}}
\newcommand{\define}{\stackrel{\triangle}{=}}
\theoremstyle{remark}
\newtheorem{rem}{Remark}

\graphicspath{./figs/}

%\bibliographystyle{ieeetr}
\begin{document}
%

\bibliographystyle{IEEEtran}


\vspace{3cm}

\title{
	%	\logo{
	Gate Assignment

	\large{EE:1205 Signals and Systems}

	Indian Institute of Technology, Hyderabad
	%	}
}
\author{Kunal Thorawade

EE23BTECH11035
}	
\maketitle


\newpage

%\tableofcontents

\bigskip
 
 \renewcommand{\thefigure}{\theenumi}
 \renewcommand{\thetable}{\arabic{table}}
 \renewcommand{\thefigure}{\arabic{figure}}
 %\renewcommand{\theequation}{\theenumi}

 \textbf{Question}:
 In the circuit shown below, the amplitudes of the voltage across the resistor and the capacitor are equal. What is the value of the angular frequency $\omega_o$ (in rad/s)? 
 (Round off the answer to one decimal place.)
 \hfill(GATE BM 32 2023)
 \begin{circuitikz}
	     % Voltage source
	     \draw (0,0) to[sV, v=$100\cos(\omega_{0} t)$] (0,2);
	         
		     % Resistor
		         \draw (0,2) to[R, l=$1\text{ k}\Omega$] (3,2);
			     
			         % Capacitor
				     \draw (3,2) to[C, l=$100\mu\text{F}$] (3,0);
				         % Ground
					     \draw (3,0) -- (0,0);
 \end{circuitikz}

 \solution
 \fi
 \begin{table}[ht]
	  \centering
	    \begin{tabular}{|c|c|c|}
		        \hline
			   \textbf{ Parameter} & \textbf{Value} & \textbf{Description} \\
			       \hline
			           $v\brak{t}$ & $100cos\brak{\omega_0 t}$ & Input Voltage \\
				       \hline
				           $R$ & $1\text{ k}\Omega$ & Resistance \\
					       \hline
					           $C$ & $100\mu\text{F}$ & Capacitance \\
						       \hline
						           $\omega_0$ & ? & Angular Frequency  \\
							       \hline
							           $Z_R = R$ & $10^3$ & Impedance for resistor  \\
								       \hline
								           $Z_C = \frac{1}{j\omega C}$ & $\frac{10^{4}}{j\omega_0}$ & Impedance for capacitor  \\
									       \hline
									           $Z = R + \frac{1}{j\omega C}$ & $10^3 + \frac{10^4}{j\omega_0}$ & Total Impedance \\
										       \hline
										         \end{tabular}
											   \vspace{2mm}
											     \caption{Parameter Table}
											       \label{BM_23_32}
\end{table}

 \begin{align}
	 R &\stackrel{\mathcal{F}}{\longleftrightarrow} R \\
	 C &\stackrel{\mathcal{F}}{\longleftrightarrow} \frac{1}{j\omega_0 C} \\
	 \abs{V_R\brak{\omega}} &= \abs{V_C\brak{\omega}} \\
	 \implies \abs{Z_R} &= \abs{Z_C} \\
	 10^3 &= \frac{10^4}{\omega_0} \\
	 \therefore \omega_0 &= 10.0
 \end{align}
 \begin{circuitikz}
	 % Voltage source
	 \draw (0,0) to[sV, v=$V(\omega)$] (0,2);
	 % Resistor
	 \draw (0,2) to[R, l=$R$, i=$I(\omega)$] (3,2);
	 % Capacitor
	 \draw (3,2) to[C, l=$\frac{1}{j\omega_0 C}$] (3,0);
	 % Ground
	 \draw (3,0) -- (0,0);
 \end{circuitikz}
 \begin{figure}[ht]
	     \centering
	         \includegraphics[width = 8cm]{2023/BM/32/figs/fig1.jpg}
		     \caption{Plot of $\abs{Z} = 10^3(\frac{\sqrt{100 + \omega_0^2}}{\omega_0}) $ }
		         \label{fig1.BM.32}
 \end{figure}
 \begin{figure}[ht]
	     \centering
	         \includegraphics[width = 8cm]{2023/BM/32/figs/fig2.png}
		     \caption{Plot of Voltage Across Capacitor and Resistor}
		         \label{fig2.BM.32}
 \end{figure}
 

\pagebreak
\item Let $ w ^{4} = 16j $. Which of the fo
    llowing can not be the value of w?\\\
    \
 52 (A)   $2e^\frac{j2 \pi}{8}$\\
 53 (B)   $2e^\frac{j \pi}{8}$\\
 54 (C)   $2e^\frac{j5 \pi}{8}$\\
 55 (D)   $2e^\frac{j9 \pi}{8}$\\
\hfill{(GATE 2023 EC)}\\               \solution                              \input{2023/EC/13/gateec131.tex}     \pagebreak
\item In the diagram shown, the frequency of the sinusoidal source voltage $V_s$ is 50 Hz.The load voltage is 230 V (RMS), and the load impedance is $\frac{230}{\sqrt{2}}$+$j\frac{230}{\sqrt{2}}$ $\Omega$. The value of attenuator $A_1$=$\frac{1}{50\sqrt{2}}$.The multiplier output voltage $V_o=\frac{V_xV_y}{1V}$, where $V_x$ and $V_y$ are the inputs. The magnitude of the average value of the multiplier output $V_0$ is \hspace{3cm}\rule{5cm}{0.4pt} V

\vspace{1cm}

\begin{circuitikz}
    \draw (0,0) to[sV, l=$V_s$] (0,3) -- (6,3) ;
    \draw [european](6,3) to[R, l=Load ](6,0)  --(2,0)  ;
     \draw (5.4,0) to  (5.1,0);
    \draw (2,0) to[R, l=$1 \Omega$]   (0,0);
    \draw (5.25,3) to (5.25,-4.75) -- (4.25,-4.75) ;
    \draw  (4.25,-4) rectangle (3.25,-5) node [midway] {$A_1$};
    \draw (0,0) node[ground]{};
    \draw (4.75,0) to (4.75,-4.35) -- (4.25,-4.35);
   \draw (3.25,-4.5) to (3,-4.5);
    \draw (3,-4) rectangle(-0.5, -5) node [midway] {$+90^\circ$ phase  
    shifter};
    \draw (2.5,-4) to (2.5,-3.5) -- (3, -3.5)-- (3,-2.5) -- (2.5, -1.6339746) --(2,-2.5) -- (2,-3.5) -- (2.5,-3.5); 
    \node at (2.25,-3.75) {$V_y$};
    \draw (2, -2.5) to (3, -3.5);
    \draw (3,-2.5) to (2,-3.5) ;
    \draw (4,0) to (4,-3.25) -- (3,-3.25);
     \node at (3.25,-3) {$V_x$};
    \draw (2.5, -1.3)   to   (2.5, -1.6339746) ;
    \node at (2.15,-1.3) {$V_o$};
\end{circuitikz}

\vspace{2cm}		
\hfill{GATE 2023 IN}	
\solution\\
\iffalse
\let\negmedspace\undefined
\let\negthickspace\undefined
\documentclass[journal,12pt,twocolumn]{IEEEtran}
\usepackage{cite}
\usepackage{amsmath,amssymb,amsfonts,amsthm}
\usepackage{algorithmic}
\usepackage{graphicx}
\usepackage{textcomp}
\usepackage{xcolor}
\usepackage{txfonts}
\usepackage{listings}
\usepackage{enumitem}
\usepackage{mathtools}
\usepackage{gensymb}
\usepackage{comment}
\usepackage[breaklinks=true]{hyperref}
\usepackage{tkz-euclide} 
\usepackage{listings}
\usepackage{gvv}  
\usepackage{circuitikz}
\usetikzlibrary{intersections}
\usepackage{tikz}
\def\inputGnumericTable{}                                 
\usepackage[latin1]{inputenc}                                
\usepackage{color}                                            
\usepackage{array}                                            
\usepackage{longtable}                                       
\usepackage{calc}                                             
\usepackage{multirow}                                         
\usepackage{hhline}                                           
\usepackage{ifthen}                                           
\usepackage{lscape}


\newtheorem{theorem}{Theorem}[section]
\newtheorem{problem}{Problem}
\newtheorem{proposition}{Proposition}[section]
\newtheorem{lemma}{Lemma}[section]
\newtheorem{corollary}[theorem]{Corollary}
\newtheorem{example}{Example}[section]
\newtheorem{definition}[problem]{Definition}
\newcommand{\BEQA}{\begin{eqnarray}}
\newcommand{\EEQA}{\end{eqnarray}}
\newcommand{\define}{\stackrel{\triangle}{=}}
\theoremstyle{remark}
\newtheorem{rem}{Remark}

\begin{document}
\bibliographystyle{IEEEtran}
\vspace{3cm}
\title{\textbf{IN-2023}}
\author{EE23BTECH11053-R.Rahul$^{*}$% <-this % stops a space
}
\maketitle
\newpage
\bigskip

\textbf{QUESTION:}\\
61. In the diagram shown, the frequency of the sinusoidal source voltage $V_s$ is 50 Hz.The load voltage is 230 V (RMS), and the load impedance is $\frac{230}{\sqrt{2}}$+$j\frac{230}{\sqrt{2}}$ $\Omega$. The value of attenuator $A_1$=$\frac{1}{50\sqrt{2}}$.The multiplier output voltage $V_o=\frac{V_xV_y}{1V}$, where $V_x$ and $V_y$ are the inputs. The magnitude of the average value of the multiplier output $V_0$ is \hspace{3cm}\rule{5cm}{0.4pt} V

\vspace{1cm}

\begin{circuitikz}
    \draw (0,0) to[sV, l=$V_s$] (0,3) -- (6,3) ;
    \draw [european](6,3) to[R, l=Load ](6,0)  --(2,0)  ;
     \draw (5.4,0) to  (5.1,0);
    \draw (2,0) to[R, l=$1 \Omega$]   (0,0);
    \draw (5.25,3) to (5.25,-4.75) -- (4.25,-4.75) ;
    \draw  (4.25,-4) rectangle (3.25,-5) node [midway] {$A_1$};
    \draw (0,0) node[ground]{};
    \draw (4.75,0) to (4.75,-4.35) -- (4.25,-4.35);
   \draw (3.25,-4.5) to (3,-4.5);
    \draw (3,-4) rectangle(-0.5, -5) node [midway] {$+90^\circ$ phase  
    shifter};
    \draw (2.5,-4) to (2.5,-3.5) -- (3, -3.5)-- (3,-2.5) -- (2.5, -1.6339746) --(2,-2.5) -- (2,-3.5) -- (2.5,-3.5); 
    \node at (2.25,-3.75) {$V_y$};
    \draw (2, -2.5) to (3, -3.5);
    \draw (3,-2.5) to (2,-3.5) ;
    \draw (4,0) to (4,-3.25) -- (3,-3.25);
     \node at (3.25,-3) {$V_x$};
    \draw (2.5, -1.3)   to   (2.5, -1.6339746) ;
    \node at (2.15,-1.3) {$V_o$};
\end{circuitikz}

\vspace{2cm}		
	
\solution
\fi
\begin{table}[h]
  \centering
  \renewcommand{\arraystretch}{1.5}
\begin{tabular}{|c|c|c|}
\hline
Parameter & Description & Value \\\hline
\( V_s \) & sinusoidal Source voltage & 230 V(RMS)\\ \hline
\(V_1 \) & voltage across attenuator &  \\\hline
\( V_x and V_y \) & inputs voltages& \\ \hline
\(A_1\) & attenuator&  $\frac{1}{50\sqrt{2}}$ \\ \hline
\(Z\) & Load Impedance& $\frac{230}{\sqrt{2}}+j\frac{230}{\sqrt{2}}$ $\Omega$ \\ \hline
\(V_0\) & output voltage & $V_0=\frac{V_xV_y}{1V} $\\ \hline
\end{tabular}
\caption{variables}
  \label{tab:xn}
\end{table}
\begin{enumerate}
    \item 

Let the curret in load be I

\begin{center}
\begin{align}
     I &=\frac{V_s(peak)}{Z}\\
     &=\frac{230\sqrt{2}}{\frac{230}{\sqrt{2}}+j\frac{230}{\sqrt{2}}} \\
     &=\sqrt{2}(1-j)
\end{align}
\end{center}
\item
voltage at attenuator 
\begin{align}
    V_1&=V_sA_1\\
    &=230\frac{1}{50\sqrt{2}}V\\
    &=\frac{4.6}{\sqrt{2}}V
\end{align}

\begin{center}
    \begin{align}
        V_y&=4.6\sin(\omega t+90^\circ)\\
        V_x&=I\times 1\Omega\\
        &=2\sqrt{2}\sin(\omega t-45^\circ)\\
        V_0&=9.2\sqrt{2}(\frac{\cos(135)-\cos(2\omega t)}{2})\\
        &=4.6-4.6\sqrt{2}\cos(2\omega t)\\
    \end{align}
\end{center}

\item 
Let f(t)= $4.6-4.6\sqrt{2}\cos(2\omega t)$
\begin{align}
    V_o<avg>&=\frac{1}{T}\int_{0}^{T}(4.6-4.6\sqrt{2}\cos(2\omega t))\,dt\\
    &=\frac{\omega}{\pi} \left[\int_{0}^{\frac{\pi}{\omega}}4.6\,dt-4.6\sqrt{2}\int_{0}^{\frac{\pi}{\omega}}\cos(2\omega t)\,dt\right]\\
    &=\frac{\omega}{\pi}\left[4.6\frac{\pi}{\omega}-4.6\sqrt{2}\left[\frac{\sin(2\pi )}{2\omega}\right]\right]\\
    &=4.6
\end{align}
\end{enumerate}
\begin{figure}[h]
      \centering
       \includegraphics[width=1\linewidth]{2023/IN/61/figs/V_x.png} % Adjust the width as needed
        \caption{$plot of V_x$}
\end{figure}
\begin{figure}[h]
      \centering
       \includegraphics[width=1\linewidth]{2023/IN/61/figs/V_y.png} % Adjust the width as needed
        \caption{$plot of V_y$}
\end{figure}
\begin{figure}[h]
      \centering
       \includegraphics[width=1\linewidth]{2023/IN/61/figs/V_o.png} % Adjust the width as needed
        \caption{$plot of V_o$}
\end{figure}
%\end{document}

\end{enumerate}
